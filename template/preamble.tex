% ====================== Pakiety =========================

% Polski
\usepackage[polish]{babel}
\usepackage[T1]{fontenc}

% Marginesy
\usepackage[a4paper, top=2.5cm, bottom=2.5cm, inner=3cm, outer=2cm]{geometry}

% Stylizacja spisu treści
\usepackage{tocloft}

% Czcionki
\usepackage{fontspec}

% Nagłówki i stopki
\usepackage{fancyhdr}

% Odstępy między liniami
\usepackage{setspace}

% Bibliografia
\usepackage[backend=biber, sorting=none]{biblatex}
\usepackage{csquotes}

% Stylizacja nagłówków
\usepackage{titlesec}
\usepackage{titletoc}

% Generowanie Lorem Ipsum
\usepackage{lipsum} 

% Wcinanie pierwszych akapitów
\usepackage{indentfirst}

% Hiperlinki
\usepackage[draft=false]{hyperref}

% Unikanie przełamania strony w akapicie
\usepackage[defaultlines=2,all]{nowidow}

% Zdjęcia
\usepackage{graphicx}
\usepackage{subcaption}

% Bloki komentarzy i bezpośredni tekst
\usepackage{verbatim}

% Stylizacja list
\usepackage{enumitem}

% Pakiet do wielu kolumn
\usepackage{multicol}

% Pakiet wspierający wiele opcjonalnych parametrów
\usepackage{xargs}

% Pakiet do umieszczania listingów kodu
\usepackage{listings}

% Kolory
\usepackage{color}

% Pakiet dotyczący języka francuskiego, ale usuwa sieroty
% \usepackage[nosingleletter]{impnattypo}

% Poprawa typesettingu
% \usepackage{microtype}

% Notatki
% \usepackage[colorinlistoftodos,prependcaption]{todonotes}

% Reprezentowanie struktury plików
\usepackage{forest}

% Opływanie zdjęć przez tekst
\usepackage{wrapfig}

% Dodatkowe narzędzia do tabel
\usepackage{makecell}

% Usuwa odstępy pomiędzy rozdziałąmi na listach obrazków
\usepackage[figurewithin=none]{caption}

% Naprawia przełamywanie linii wewnątrz \texttt
\usepackage[htt]{hyphenat}

% Dodaje możliwość używania zaawansowanych obramowań w tabelach
\usepackage{hhline}

\usepackage{array}
\usepackage{multirow}

\newcolumntype{R}[1]{>{\raggedleft\let\newline\\\arraybackslash\hspace{0pt}}m{#1}}

% ================== Konfiguracja =====================

% Ustawienie domyślnych czcionek
\setmainfont{Times New Roman}
\setsansfont{Arial}

% Odstęp między liniami
\setstretch{1.5}

% Głębokość wcięcie akapitów
\setlength\parindent{0cm}

% Usunięcie odstępu po tytule w spisie treści
\addtocontents{toc}{\vspace{-5ex}}

% Nagłówek strony
\pagestyle{fancy}
\fancyhf{}
\rhead{Tłumaczenie zapytań w języku polskim na SQL}
\lhead{inż. Łukasz Klimkiewicz}
\fancyfoot[RO,LE]{\thepage} % strona tytułowa drukowana jednostronnie

% Zapewnienie ilości miejsca na nagłówek
\setlength{\headheight}{14.5pt}

% Dodanie nagłówka na każdej stronie
\fancypagestyle{plain}{}

% Usunięcie domyślnego tytułu spisu treści
\makeatletter
\renewcommand{\@cftmaketoctitle}{}
\makeatother

% Usunięcie domyślnego tytułu spisu zdjęć
\makeatletter
\renewcommand{\@cftmakeloftitle}{}
\makeatother

% Usunięcie domyślnego tytułu spisu tabel
\makeatletter
\renewcommand{\@cftmakelottitle}{}
\makeatother

% Zmniejsza automatycznie dodawane spacje między paragrafami
\setlength{\parskip}{3mm plus 1.0pt minus0mm}

% Usunięcie przerw przed rozdziałami i sekcjami
% Usunięcie przerw przez rozdziałami
\titlespacing{\chapter}{0pt}{-30pt}{0mm} % -\parskip
\titlespacing{\section}{0pt}{5mm}{0mm}
\titlespacing{\subsection}{0pt}{5mm}{0mm}

% Dodanie poprawnego numerowanie rozdziałów
\titleformat{\chapter}[hang] 
{\normalfont\LARGE\bfseries}{\thechapter}{1em}{} 

\titleformat*{\chapter}{\bfseries\LARGE\sffamily}
\titleformat*{\section}{\bfseries\Large\sffamily}
\titleformat*{\subsection}{\bfseries\large\sffamily}

% Dodanie bibliografii
\addbibresource{bibliography/mendeley.bib}
\addbibresource{bibliography/others.bib}

% Konfiguracja hiperlinków
\hypersetup{
    colorlinks,
    citecolor=black,
    filecolor=black,
    linkcolor=black,
    urlcolor=black
}

% Ograniczenie głębokości spisu treści
\setcounter{tocdepth}{1}

% Usunięcie odstępu pomiędzy rozdział w spisie treści
\makeatletter
\renewcommand*\l@chapter[2]{%
  \ifnum \c@tocdepth >\z@
    \addpenalty\@secpenalty
    \setlength\@tempdima{1.5em}%
    \begingroup
      \parindent \z@ \rightskip \@pnumwidth
      \parfillskip -\@pnumwidth
      \leavevmode \bfseries
      \advance\leftskip\@tempdima
      \hskip -\leftskip
      #1\nobreak\hfil \nobreak\hb@xt@\@pnumwidth{\hss #2}\par
    \endgroup
  \fi}
\makeatother

% Usunięcie odstępu między elementami list i górnego odstępu
\setlist{nolistsep, topsep=0pt}

% Dostosowanie odstępu przed i po kolumnach
\setlength{\multicolsep}{3pt plus 0pt minus 0pt}

% Odstęp pomiędzy ramką, a zawartośc\setlength{\fboxsep}{0pt}

% Definicja kolorów
\definecolor{lightgray}{rgb}{.95,.95,.92}
\definecolor{purple}{rgb}{1, 0, 1}
\definecolor{mediumorchid}{rgb}{0.72, 0.33, 0.82}
\definecolor{darkviolet}{rgb}{0.58, 0, 0.82}
\definecolor{green}{rgb}{0, 0.5, 0}

% Domyślna długość lorem ipsum
\SetLipsumDefault{2-6}

% Domyślnie nagłóœki
% \captionsetup[subfigure]{justification=centering}
\captionsetup[table]{name=Tabela}

% Strony nie muszą być równej wysokości
\raggedbottom

\colorlet{punct}{red!60!black}
\definecolor{background}{HTML}{EEEEEE}
\definecolor{delim}{RGB}{20,105,176}
\colorlet{numb}{magenta!60!black}

\definecolor{delim}{RGB}{20,105,176}
\definecolor{numb}{RGB}{106, 109, 32}
\definecolor{string}{rgb}{0.64,0.08,0.08}

% Określenie stylów listingów
\lstdefinelanguage{json}{
    morestring=[b]",
    stringstyle=\color{string},
    literate=
     *{0}{{{\color{numb}0}}}{1}
      {1}{{{\color{numb}1}}}{1}
      {2}{{{\color{numb}2}}}{1}
      {3}{{{\color{numb}3}}}{1}
      {4}{{{\color{numb}4}}}{1}
      {5}{{{\color{numb}5}}}{1}
      {6}{{{\color{numb}6}}}{1}
      {7}{{{\color{numb}7}}}{1}
      {8}{{{\color{numb}8}}}{1}
      {9}{{{\color{numb}9}}}{1}
      {\{}{{{\color{delim}{\{}}}}{1}
      {\}}{{{\color{delim}{\}}}}}{1}
      % {[}{{{\color{delim}{[}}}}{1}
      % {]}{{{\color{delim}{]}}}}{1},
}

\lstdefinelanguage{SQL}{
  morekeywords={select,from,where,inner,join,on,order,by,group,asc,desc,limit,and,or,not,is,null,like,in,between,as},
  sensitive=false,
  morecomment=[l]{--},
  morecomment=[s]{/*}{*/},
  morestring=[b]",
}

\lstset{
   backgroundcolor=\color{lightgray},
   extendedchars=true,
   basicstyle=\footnotesize\ttfamily,
   showstringspaces=false,
   showspaces=false,
   numbers=left,
   numberstyle=\small,
   numbersep=9pt,
   tabsize=2,
   breaklines=true,
   showtabs=false,
   captionpos=b,
   frame=single,
   upquote=true,
   aboveskip=2em,
}

\lstset{
   language=SQL,
   keywordstyle=\color{blue}\bfseries,
   commentstyle=\color{darkgreen},
   stringstyle=\color{red},
   showstringspaces=false,
   breaklines=true,
   postbreak=\mbox{\textcolor{red}{$\hookrightarrow$}\space}
}

% ====================== Nowe komendy =========================

% Dodanie komendy do zakreślania
\newcommand{\todolater}[1]{\todo[color=yellow,inline]{TODO: #1}}

% Komenda unsure
\newcommandx{\unsure}[2][1=]{\todo[linecolor=red,backgroundcolor=red!25,bordercolor=red,#1]{#2}}

% Fake chapter & fake section
\newcommand{\fchapter}[1]{\chapter{#1}\lipsum[3]}
\newcommand{\ftchapter}[2]{\chapter{#1}\todolater{#2}\lipsum[3]}

\newcommand{\fsection}[1]{\section{#1}\lipsum[3]}
\newcommand{\ftsection}[2]{\section{#1}\todolater{#2}\lipsum[3]}

% Komenda do towrzenia wykazu skrótów
\newcommand{\abbrev}[2]{\item \textbf{#1} - {#2}}

\newcommand{\filename}[1]{\texttt{#1}}

\definecolor{code_color}{RGB}{138,0,0}
\newcommand{\code}[1]{\texttt{\color{code_color}{#1}}}

\newcommand{\sql}[1]{\texttt{\textcolor{code_color}{#1}}}

\renewcommand\theadfont{\bfseries\sffamily}

\newcommand{\s}{\hphantom{0}}

\newcommand{\bibtitle}[1]{\textit{\enquote{#1}}}

