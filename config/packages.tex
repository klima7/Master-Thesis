% Polski
\usepackage[polish]{babel}
\usepackage[T1]{fontenc}

% Marginesy
\usepackage[a4paper, top=2.5cm, bottom=2.5cm, inner=3cm, outer=2cm, footskip=1.25cm]{geometry}

% Stylizacja spisu treści
\usepackage{tocloft}

% Czcionki
\usepackage{fontspec}

% Nagłówki i stopki
\usepackage{fancyhdr}

% Odstępy między liniami
\usepackage{setspace}

% Bibliografia
\usepackage{csquotes}
\usepackage[backend=biber,style=authoryear,mincitenames=1,maxcitenames=1]{biblatex}

% Stylizacja nagłówków
\usepackage{titlesec}
\usepackage{titletoc}

% Generowanie Lorem Ipsum
\usepackage{lipsum} 

% Wcinanie pierwszych akapitów
\usepackage{indentfirst}

% Hiperlinki
\usepackage[draft=false,hidelinks,breaklinks]{hyperref}

% Unikanie przełamania strony w akapicie
\usepackage[defaultlines=2,all]{nowidow}

% Zdjęcia
\usepackage{graphicx}
\usepackage{subcaption}

% Bloki komentarzy i bezpośredni tekst
\usepackage{verbatim}

% Stylizacja list
\usepackage{enumitem}

% Pakiet do wielu kolumn
\usepackage{multicol}

% Pakiet wspierający wiele opcjonalnych parametrów
\usepackage{xargs}

% Pakiet do umieszczania listingów kodu
\usepackage{listings}

% Kolory
\usepackage{color}

% Pakiet dotyczący języka francuskiego, ale usuwa sieroty
\usepackage[nosingleletter]{impnattypo}

% Poprawa typesettingu
\usepackage{microtype}

% Notatki
% \usepackage[colorinlistoftodos,prependcaption]{todonotes}

% Reprezentowanie struktury plików
\usepackage{forest}

% Opływanie zdjęć przez tekst
\usepackage{wrapfig}

% Dodatkowe narzędzia do tabel
\usepackage{makecell}

% Usuwa odstępy pomiędzy rozdziałąmi na listach obrazków i zmienia format nagłówków
\usepackage[figurename=Rys.,font=footnotesize,labelfont=bf]{caption}

% Naprawia przełamywanie linii wewnątrz \texttt
\usepackage[htt]{hyphenat}

% Dodaje możliwość używania zaawansowanych obramowań w tabelach
\usepackage{hhline}

% Dodaje dodatki do tabel
\usepackage{array}
\usepackage{multirow}

% Dodaje więcej rozmiarów czcionki
\usepackage[12pt]{moresize}

% Dodaje wykresy
\usepackage{tikz}
\usepackage{pgfplots}

% Dodaje możliwość wykonywania obliczeń w latex
\usepackage{expl3}[2012-07-08]

% Dodaje bardziej elastyczne ustalanie pozycji
\usepackage{float}

% Dodaje appendixy
\usepackage[toc]{appendix}  % toc,page - jeżli chcemy stronę "Appendix" przed

% Pakiet potrzebny do automatycznego sortowania
\usepackage{datatool}

% Pakiet do wyświetlania liczb
\usepackage{numprint}

% Do plików svg
\usepackage[inkscapearea=page]{svg}
