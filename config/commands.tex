\definecolor{code_color}{RGB}{138,0,0}

% Komenda do zakreślania
\newcommand{\todolater}[1]{\todo[color=yellow,inline]{TODO: #1}}

% Komenda do zaznaczanie niepewnych fragmentów
\newcommandx{\unsure}[2][1=]{\todo[linecolor=red,backgroundcolor=red!25,bordercolor=red,#1]{#2}}

% Komendy do tworzenia sztucznych rodziałów i sekcji
\newcommand{\fchapter}[1]{\chapter{#1}\lipsum[3]}
\newcommand{\ftchapter}[2]{\chapter{#1}\todolater{#2}\lipsum[3]}
\newcommand{\fsection}[1]{\section{#1}\lipsum[3]}
\newcommand{\ftsection}[2]{\section{#1}\todolater{#2}\lipsum[3]}

\newcommand{\s}{\hphantom{0}}

\newcommand{\bibtitle}[1]{\textit{\enquote{#1}}}

\newcommand{\code}[1]{\texttt{\color{code_color}{#1}}}

\newcommand{\sql}[1]{\texttt{\textcolor{code_color}{#1}}}

% Komenda do tworzenia wykazu skrótów
\newcommand{\abbrev}[2]{\item \textbf{#1} - {#2}}

\newcommand{\filename}[1]{\texttt{#1}}

% Komenda dodająca źródło do podpisów
% https://tex.stackexchange.com/questions/95029
\newcommand{\source}[1]{\caption*{\small \hfill Źródło: {#1}} }

% Komendy używane do tabelek z wynikami
\newcommand{\slashsep}{\textcolor{red}{\textbf{/}}}
\newcommand{\threevals}[3]{\small{#1}\slashsep\small{#2}\slashsep\small{#3}}
\newcommand{\twovals}[2]{{#1} \slashsep{ #2}}