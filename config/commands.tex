% Kolory
\definecolor{code_color}{RGB}{138,0,0}
% \definecolor{citation_color}{RGB}{0,51,204}
\definecolor{citation_color}{RGB}{138,0,0}

% Komenda do zaznaczania rzeczy do zrobienia później
\newcommand{\todolater}[1]{\todo[color=yellow,inline]{TODO: #1}}

% Komenda do zaznaczanie niepewnych fragmentów
\newcommandx{\unsure}[2][1=]{\todo[linecolor=red,backgroundcolor=red!25,bordercolor=red,#1]{#2}}

% Komendy do tworzenia sztucznych rodziałów i sekcji
\newcommand{\fchapter}[1]{\chapter{#1}\lipsum[3]}
\newcommand{\ftchapter}[2]{\chapter{#1}\todolater{#2}\lipsum[3]}
\newcommand{\fsection}[1]{\section{#1}\lipsum[3]}
\newcommand{\ftsection}[2]{\section{#1}\todolater{#2}\lipsum[3]}

% Komenda robiąca puste miejsce
\newcommand{\s}{\hphantom{0}}

% Stylowanie do tytułów z bibliografii
\newcommand{\bibtitle}[1]{\textit{#1}}

% Komendy do formatowania fragmentów tekstów
\newcommand{\code}[1]{\begin{hyphenrules}{nohyphenation}\texttt{\color{code_color}{#1}}\end{hyphenrules}}
\newcommand{\hcode}[1]{\begin{hyphenrules}{nohyphenation}{\color{code_color}{#1}}\end{hyphenrules}}
\newcommand{\sql}[1]{\texttt{\textcolor{code_color}{#1}}}

% Komenda do linku w stópce
\newcommand{\furl}[1]{\footnote{\url{#1}}}

% Komenda do tworzenia wykazu skrótów
\newcommand{\abbrev}[2]{\sortitem{\textbf{#1} - {#2}}}

% Komenda do długich nagłówków
\newcommand{\lcaption}[2]{\caption[#1]{#1. \textit{#2}}}

% Komenda dodająca źródło do podpisów
\newcommand{\source}[1]{\caption*{\footnotesize \hfill Źródło: {#1}} }

% Komendy używane do tabelek z wynikami
\newcommand{\slashsep}{\textcolor{red}{\textbf{/}}}
\newcommand{\threevals}[3]{\small{#1}\slashsep\small{#2}\slashsep\small{#3}}
\newcommand{\twovals}[2]{{#1} \slashsep{ #2}}

% Zakreślanie tekstu w prompcie
\newcommand{\highprompt}[1]{\colorbox{yellow}{#1}}

% Cytowanie
\newcommand{\mycite}[1]{({\color{citation_color}\cite{#1}})}

% Komendy do automatycznie sortujących się list
\newcommand{\sortitem}[2][\relax]{%
  \DTLnewrow{list}
  \ifx#1\relax
    \DTLnewdbentry{list}{sortlabel}{#2}
  \else
    \DTLnewdbentry{list}{sortlabel}{#1}
  \fi%
  \DTLnewdbentry{list}{description}{#2}
}
\newenvironment{sortedlist}{%
  \DTLifdbexists{list}{\DTLcleardb{list}}{\DTLnewdb{list}}
}{%
  \DTLsort*{sortlabel}{list}
  \begin{itemize}%
    \DTLforeach*{list}{\theDesc=description}{%
      \item \theDesc}
  \end{itemize}%
}

% Komenda zabraniająca przełamywania strony
\makeatletter 
\newcommand\nobreakpar{\par\nobreak\@afterheading} 
\makeatother