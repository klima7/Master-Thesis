\chapter{Finalne rozwiązanie}
Wszystkie przeprowadzone we wcześniejszej części testy modeli sprowadzały się wyłącznie do oceny automatycznej. Wyciągnięto z niej wniosek, że bardzo dobrze spisuje się model \code{RESDSQL}, a \code{C3} również jest godny uwagi. Z tego względu w niniejszym rozdziale postanowiono dokonać ponownego treningu \code{RESDSQL}, lecz z wykorzystaniem większej wersji modelu \code{T5} oraz większej ilości danych. Tak jak wcześniej dokonano automatycznej oceny, lecz poza tym zaimplementowano interfejs graficzny, pozwalający na praktyczne wykorzystanie oraz manualną ocenę dwóch wybranych modeli. Stworzoną aplikację przedstawiono kilku osobom w celu przetestowania i wyrażenia swojej opinii. Te skrótowo opisane działania zostaną w kolejnych sekcjach rozwinięte.

\section{Ponowny trening RESDSQL}
Jako że wyniki modelu \code{RESDSQL} okazały się najlepsze spośród wszystkich klasycznych, trenowalnych rozwiązań, postanowiono dokonać jego nauki ponownie, lecz z wykorzystaniem rozszerzonego zbioru oraz większego wariantu modelu \code{T5}, licząc na jeszcze lepsze rezultaty.

Wcześniej do treningu wykorzystano jedynie warianty zbioru \code{Spider}, ponieważ pozwoliło to skrócić czas treningu oraz wykonywać rzetelne porównania z modelami tworzonymi przez innych badaczy, którzy również dokonywali treningu wyłącznie na tym zbiorze. Teraz poza zbiorem \code{Spider} do danych treningowych postanowiono włączyć \code{Spider-Syn}, \code{SParC} oraz \code{CoSQL} w wariantach z polskimi i angielskimi schematami baz danych, a także ich angielskie pierwowzory. Zbiór \code{Spider-DK} pominięto, ponieważ zawiera on jedynie przykłady testowe. Posługując się wcześniej wprowadzonymi nazwami, można powiedzieć inaczej, że do zbioru treningowego wybrano \code{Mix-Spider}, \code{Mix-Spidersyn}, \code{Mix-Sparc} oraz \code{Mix-Cosql}. W ten sposób ilość danych treningowych została zwiększona prawie dwukrotnie.

Wariant modelu \code{T5} postanowiono zmienić z wcześniej wykorzystanego \code{mt5-base} na \code{mt5-large}, który zawiera dwukrotnie większą liczbę parametrów. Powoduje ona, że model jest w stanie pomieścić znacznie większą wiedzę. W fazie eksperymentów nie skorzystano z tego, ponieważ nie dysponowano kartą graficzną o odpowiednio dużej ilości pamięci \code{VRAM}. \textbf{Stało się to możliwe dzięki uprzejmości p. Grzegorza Warzechy z firmy TaskPilot, który udostępnił odpowiednie zasoby sprzętowe} i do treningu wykorzystano potężną kartę \code{Nvidia RTX 4090}, dysponującą 24 GB pamięci \code{VRAM}.

Na rozwiązanie \code{RESDSQL}, jak opisywano wcześniej, składają się dwa modele. Pierwszego, oceniającego istotność tabel i kolumn, nie uczono ponownie, lecz wykorzystano ten wytrenowany wcześniej. Zamiana rozmiaru modelu językowego \code{T5} i ponowny trening tyczy się jedynie drugiego etapu, w którym następuje ostateczne generowanie zapytań SQL. Naukę kontynuowano przez 14 godzin, a jej wyniki przedstawiono w tabeli \ref{tab:resdsql-final-difficulty}.

\begin{table}[H]
    \centering
    \begin{tabular}{|l|r|r|r|r|r|}
        \hline
        \thead{Zbiór} & \thead{Easy} & \thead{Medium} & \thead{Hard} & \thead{Extra Hard} & \thead{Razem} \\
        \hline
        Pol-Spider & 
        \threevals{85,5}{76,8}{87,7} &
        \threevals{73,4}{66,6}{78,8} &
        \threevals{52,6}{48,3}{66,7} &
        \threevals{41,6}{38,0}{57,2} &
        \threevals{67,7}{61,4}{75,4} \\
        
        Pol-Spider-PL &
        \threevals{86,7}{78,6}{87,9} &
        \threevals{74,7}{67,9}{79,4} &
        \threevals{54,0}{48,9}{67,8} &
        \threevals{41,0}{36,7}{56,6} &
        \threevals{68,7}{62,3}{75,8} \\
        
        Pol-Spider-EN &
        \threevals{84,3}{75,0}{87,5} &
        \threevals{72,2}{65,2}{78,3} &
        \threevals{51,1}{47,7}{65,5} &
        \threevals{42,2}{39,2}{57,8} &
        \threevals{66,7}{60,4}{75,0} \\
        
        En-Spider &
        \threevals{86,7}{83,5}{88,3} &
        \threevals{76,9}{73,8}{79,6} &
        \threevals{59,2}{56,9}{65,5} &
        \threevals{47,6}{47,0}{53,0} &
        \threevals{71,6}{69,0}{75,0} \\
        
        \hline
        
        Pol-Spidersyn &
        \threevals{72,2}{64,0}{80,4} &
        \threevals{64,3}{60,3}{75,4} &
        \threevals{46,0}{45,6}{60,7} &
        \threevals{38,0}{35,5}{51,7} &
        \threevals{58,7}{54,7}{70,2} \\
        
        Pol-Spiderdk &
        \threevals{63,2}{58,2}{67,7} &
        \threevals{49,2}{44,1}{56,1} &
        \threevals{33,1}{33,1}{48,6} &
        \threevals{22,9}{21,4}{35,7} &
        \threevals{44,7}{41,0}{53,5} \\
        
        Pol-Sparc &
        \threevals{77,5}{72,1}{82,7} &
        \threevals{50,0}{44,2}{66,0} &
        \threevals{20,0}{20,0}{50,0} &
        \threevals{43,8}{43,8}{50,0} &
        \threevals{66,7}{61,5}{76,0} \\
        
        Pol-Cosql &
        \threevals{73,8}{67,9}{80,0} &
        \threevals{61,0}{54,2}{68,6} &
        \threevals{27,9}{27,9}{52,9} &
        \threevals{29,4}{26,5}{50,0} &
        \threevals{60,4}{55,4}{70,9} \\
        
        \hline
    \end{tabular}
    \lcaption{Wyniki finalnego modelu \code{RESDSQL} na poszczególnych zbiorach}{Wartości w każdej komórce posiadają format \mbox{EM Without Values / EM / EX}.}
    \label{tab:resdsql-final-difficulty}
\end{table}

\begin{figure}[H]
  \centering
  \includesvg[width=1.0\linewidth]{images/final_resdsql_radar}
  \lcaption{...}{...}
  \label{fig:final-resdsql-radar}
\end{figure}


Wyniki testów pokazują, że zwiększenie rozmiaru modelu \code{T5} wraz z rozszerzeniem zbioru treningowego pozwoliło wyraźnie podnieść wyniki wszystkich metryk w porównaniu do wcześniej trenowanej wersji. W przypadku każdego ze zbiorów testowych nastąpiło zwiększenie wartości metryki \code{EM} o przynajmniej 8 punktów procentowych. Wartość metryki \code{EX} również wzrosła, co sprawiło, że ta wersja modelu \code{RESDSQL} dorównuje pod tym kątem nawet modelowi \code{C3}.

\section{Aplikacja webowa}
W celu umożliwienia praktycznego wykorzystania oraz manualnej weryfikacji modeli \code{RESDSQL} oraz \code{C3} stworzony został interfejs graficzny w postaci aplikacji webowej. W szczególności duże wyzwanie stanowiło połączenie go z ze wspomnianymi modelami, ponieważ zostały stworzone do wykonywania predykcji w sposób wsadowy na statycznych zbiorach. W tym przypadku predykcje wykonywane musiały być natomiast dla pojedynczych pytań oraz dowolnych baz danych.

\subsection{Interfejs graficzny}
Interfejs graficzny, którego wygląd przedstawiono na rysunkach w dodatku \ref{chapter:gui}, został stworzony z wykorzystaniem biblioteki \code{streamlit}. Jej zaletą jest niebywała prostota, lecz wiąże się ona z pewnym brakiem elastyczności -- wpływanie na wygląd poszczególnych komponentów oraz interakcje pomiędzy nimi jest bardzo ograniczone. Przygotowana aplikacja wydaje się dość skomplikowana jak na możliwości tej biblioteki, lecz po dokładnym przemyśleniu struktury i szczegółowym zapoznaniem z dokumentacją udało się ją zrealizować. Na aplikację składają się trzy zakładki: pierwsza pozwala na wybór bazy danych, druga na podanie naturalnych odpowiedników dla nazw tabel i kolumn, a trzecia na generowanie zapytań SQL.

\textbf{Pierwsza zakładka} umożliwia wybór bazy danych poprzez załadowanie jej w formacie \code{sqlite}, wybranie jednej z przygotowanych baz przykładowych lub podanie skryptu SQL, który ją tworzy. Po dokonaniu decyzji ukazuje się diagram wybranej bazy, do zaimplementowania którego wykorzystano bibliotekę języka Python o nazwie \code{eralchemy}, która pozwala w łatwy sposób tworzyć diagramy relacji encji (ang. entity relation diagram). Dzięki temu możliwe jest powierzchowne zapoznanie się ze strukturą bazy i uświadomienie pytań, które można podawać do obsługującego ją interfejsu tekstowego.

\textbf{Druga zakładka} pozwala na podanie naturalnych odpowiedników dla nazw tabel i kolumn. Jej obecność jest wymuszona przez strukturę zbioru \code{Spider}, który zawiera taką informację oraz fakt, że rozwiązania \code{RESDSQL} oraz \code{C3} z niej korzystają. Jest to pewnego rodzaju informacja dodatkowa, której podawanie może być irytujące, szczególnie w przypadku dużych baz danych. Aby to usprawnić opracowano prosty algorytm, który wstępnie proponuje nazwy naturalne poprzez rozbicie identyfikatorów tabel i kolumn na poszczególne słowa. W dużej mierze ogranicza to działania użytkownika do ich weryfikacji.

\textbf{Trzecia zakładka} umożliwia zadawanie pytań do określonej wcześniej bazy danych. Ma to postać chatu, w którym naprzemiennie użytkownik podaje swoje żądania oraz model wysyła przewidziane zapytania SQL. Forma ta może być nieco myląca, ponieważ sugeruje, że wykorzystywane modele działają w sposób kontekstowy, czyli biorą pod uwagę wcześniejszą historię konwersacji, podczas gdy w rzeczywistości tak nie jest. Jest ona jednak wygodna, gdyż większość użytkowników jest z nią dobrze zaznajomiona. Generowane zapytania są natychmiastowo wykonywane i interaktywne tabele z odczytanymi rekordami są bezpośrednio pod nimi zamieszczane, co jest bardzo ważne pod kątem praktycznego wykorzystania. Aktywny w danej chwili model można zmieniać za pomocą listy rozwijanej. Jeżeli wybrany został \code{C3} konieczny jest jednak dostęp do klucza API \code{OpenAI}, który można wprowadzić w odpowiednim polu tekstowym w interfejsie. Przewidziano również możliwość uruchomienia całej aplikacji z danym kluczem, a wówczas podawanie go przez poszczególnych użytkowników nie jest konieczne. Dostęp do interfejsu chroniony jest wówczas hasłem, by przeciwdziałać naliczaniu kosztów przez nieautoryzowane osoby.

\subsection{Połączenie interfejsu z modelami}
Dość kłopotliwe okazało się połączenie stworzonego interfejsu z modelami, w szczególności z \code{RESDSQL}. W przypadku obu rozwiązań istniejący kod nie pozwala bowiem na proste generowanie zapytań do dowolnych baz danych. Wymagają, aby przykłady testowe były dostarczane w formacie wykorzystywanym przez zbiór \code{Spider}. Z tego powodu posiadając wybraną przez użytkownika bazę oraz naturalne odpowiedniki nazw konieczne okazało się każdorazowe generowanie tymczasowego, jednoelementowego zbioru danych, co wymagało dodatkowego wysiłku.

Innego rodzaju problemem okazało się generowanie przez \code{RESDSQL} zapytań w sposób, który można nazwać wsadowym. Odbywa się ono bowiem w dwóch etapach z których każdy korzysta z osobnej sieci neuronowej. Istniejący kod działa w ten sposób, że ładuje do pamięci model pierwszy i przepuszcza przez niego wszystkie dane, a następnie zastępuje go drugim i podaje do niego wszystkie wyniki pośrednie. Sprawdza się to w przypadku generowania dużej liczby zapytań SQL jednocześnie, lecz niekoniecznie dla serii pojedynczych zapytań, tak jak w rozpatrywanym przypadku. Okazuję się wówczas, że większość czasu poświęcana jest naprzemiennemu ładowaniu sieci neuronowych do pamięci zamiast rzeczywistym obliczeniom. Z tego powodu zmodyfikowano kod w taki sposób, aby w pamięci znajdywały się przez cały czas obie sieci. Oznacza to oczywiście wykorzystanie znacznie większej ilości pamięci, czemu postanowiono przeciwdziałać poprzez załadowanie parametrów modeli z wykorzystaniem precyzji 16-bitowej zamiast oryginalnie 32-bitowej. Dzięki temu wymaganie pamięciowe zostało zredukowane do 10 GB pamięci \code{VRAM}, czy też \code{RAM} w przypadku braku odpowiedniej karty graficznej, lecz zwiększa to czas generowania zapytań z kilku do kilkunastu sekund.

\section{Manualna ocena}
\lipsum[1-3]