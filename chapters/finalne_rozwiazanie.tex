\chapter{Finalne rozwiązanie}
Wszystkie przeprowadzone we wcześniejszej części testy modeli sprowadzały się wyłącznie do oceny automatycznej. Wyciągnięto z niej wniosek, że bardzo dobrze spisuję się model \code{RESDSQL}, a \code{C3} także jest godny uwagi. Z tego względu w niniejszym rozdziale postanowiono dokonać ponownego treningu \code{RESDSQL}, lecz z wykorzystaniem większej wersji modelu \code{T5} oraz z wykorzystaniem większej ilości danych. Tak jak wcześniej dokonano automatycznej oceny, lecz poza tym zaimplementowano interfejs graficzny pozwalający na praktyczne wykorzystanie oraz manualną ocenę tego rozwiązania, a także \code{C3}. Stworzoną aplikację przedstawiono kilku osobom w celu przetestowania i wyrażenia swojej opinii. Te skrótowo przedstawione działania zostaną w dalszej części dokładnie opisane.

\section{Ponowny trening RESDSQL i automatyczna ocena}
Jako, że wyniki modelu \code{RESDSQL} okazały się najlepsze spośród wszystkich klasycznych, trenowalnych rozwiązań, postanowiono dokonać jego nauki ponownie, lecz z wykorzystaniem rozszerzonego zbioru oraz większego wariantu modelu \code{T5} w nadziei na zwiększenie skuteczności.

Wcześniej do treningu wykorzystano jedynie warianty zbioru \code{Spider}, ponieważ pozwalało to skrócić czas treningu oraz wykonywać rzetelne porównania z modelami tworzonymi przez innych badaczy, którzy również dokonywali treningu wyłącznie na tym zbiorze. Poza zbiorem \code{Spider} do danych treningowych postanowiono włączyć \code{Spider-Syn}, \code{SParC} oraz \code{CoSQL} w wariancie z polskimi i angielskimi schematami baz danych, a także ich angielskie pierwowzory. Zbiór \code{Spider-DK} pominięto, ponieważ zawiera on jedynie przykłady testowe. W ten sposób ilość danych treningowych została zwiększona prawie dwukrotnie.

Wariant modelu \code{T5} postanowiono zmienić z wcześniej wykorzystanego \code{mt5-base} na \code{mt5-large}, który zawiera dwukrotnie większą liczbę parametrów. Powoduje ona, że model jest w stanie pomieścić znacznie większą wiedzę. W fazie eksperymentów nie skorzystano z tego, ponieważ nie dysponowano kartą graficzną o odpowiednio dużej ilości pamięci \code{VRAM}. \textbf{Stało się to możliwe dzięki uprzejmości Grzegorza Warzecha z firmy TaskPilot, który udostępnił odpowiednie zasoby sprzętowe}. Do treningu wykorzystano potężną kartę \code{RTX-4090}.

Na rozwiązanie \code{RESDSQL}, jak opisywano wcześniej, składają się dwa modele. Pierwszego, oceniającego istotność tabel i kolumn, nie uczono ponownie, lecz wykorzystano ten wytrenowany wcześniej. Zamiana rozmiaru modelu językowego \code{T5} i ponowny trening tyczy się jedynie drugiego etapu, w którym następuje ostateczne generowanie zapytań SQL. Naukę kontynuowano przez 14 godzin, a jego wyniki przedstawiono w tabeli \ref{tab:resdsql-final-difficulty}.

\begin{table}[H]
    \centering
    \begin{tabular}{|l|r|r|r|r|r|}
        \hline
        \thead{Zbiór} & \thead{Easy} & \thead{Medium} & \thead{Hard} & \thead{Extra} & \thead{Razem} \\
        \hline
        pol-spider & 
        \threevals{85,5}{76,8}{87,7} &
        \threevals{73,4}{66,6}{78,8} &
        \threevals{52,6}{48,3}{66,7} &
        \threevals{41,6}{38,0}{57,2} &
        \threevals{67,7}{61,4}{75,4} \\
        
        pol-spider-pl &
        \threevals{86,7}{78,6}{87,9} &
        \threevals{74,7}{67,9}{79,4} &
        \threevals{54,0}{48,9}{67,8} &
        \threevals{41,0}{36,7}{56,6} &
        \threevals{68,7}{62,3}{75,8} \\
        
        pol-spider-en &
        \threevals{84,3}{75,0}{87,5} &
        \threevals{72,2}{65,2}{78,3} &
        \threevals{51,1}{47,7}{65,5} &
        \threevals{42,2}{39,2}{57,8} &
        \threevals{66,7}{60,4}{75,0} \\
        
        en-spider &
        \threevals{86,7}{83,5}{88,3} &
        \threevals{76,9}{73,8}{79,6} &
        \threevals{59,2}{56,9}{65,5} &
        \threevals{47,6}{47,0}{53,0} &
        \threevals{71,6}{69,0}{75,0} \\
        
        \hline
        
        pol-spidersyn &
        \threevals{72,2}{64,0}{80,4} &
        \threevals{64,3}{60,3}{75,4} &
        \threevals{46,0}{45,6}{60,7} &
        \threevals{38,0}{35,5}{51,7} &
        \threevals{58,7}{54,7}{70,2} \\
        
        pol-spiderdk &
        \threevals{63,2}{58,2}{67,7} &
        \threevals{49,2}{44,1}{56,1} &
        \threevals{33,1}{33,1}{48,6} &
        \threevals{22,9}{21,4}{35,7} &
        \threevals{44,7}{41,0}{53,5} \\
        
        pol-sparc &
        \threevals{77,5}{72,1}{82,7} &
        \threevals{50,0}{44,2}{66,0} &
        \threevals{20,0}{20,0}{50,0} &
        \threevals{43,8}{43,8}{50,0} &
        \threevals{66,7}{61,5}{76,0} \\
        
        pol-cosql &
        \threevals{73,8}{67,9}{80,0} &
        \threevals{61,0}{54,2}{68,6} &
        \threevals{27,9}{27,9}{52,9} &
        \threevals{29,4}{26,5}{50,0} &
        \threevals{60,4}{55,4}{70,9} \\
        
        \hline
    \end{tabular}
    \caption{Wyniki finalnego modelu \code{RESDSQL} na poszczególnych zbiorach. Wartości w każdej komórce posiadają format EM \slashsep{ EM with values} \slashsep{ EX}}
    \label{tab:resdsql-final-difficulty}
\end{table}

Wyniki testów pokazują, że zwiększenie rozmiaru modelu \code{T5} wraz z rozszerzeniem zbioru treningowego pozwoliło wyraźnie podnieść wyniki wszystkich metryk w porównaniu do wcześniej trenowanej wersji. W przypadku każdego ze zbiorów testowych nastąpiło zwiększenie wartości metryki \code{EM} o przynajmniej 8 punktów procentowych. Wartoś metryki \code{EX} również wzrosła, co sprawiło, że ta wersja modelu \code{RESDSQL} dorównuję pod tym kątem nawet modelowi \code{C3}.

\section{Implementacja interfejsu graficznego}
W celu umożliwienia

\section{Manualna ocena}
