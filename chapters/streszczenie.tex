% ==================== polski ==================== 
\section*{Streszczenie}
\vspace{1cm}

Niniejsza praca dyplomowa dotyczy istotnego obecnie problemu tłumaczenia zapytań z języka naturalnego na SQL. Wyróżnia się wykorzystaniem polskiego w roli języka naturalnego, co do tej pory nie zyskało zainteresowania. W jej ramach stworzono odpowiednie zbiory danych, przeprowadzono eksperymenty i opracowano użyteczną aplikację.

Pracę rozpoczęto od przeanalizowania istniejących zbiorów danych ze szczególnym zwróceniem uwagi na zbiór \code{Spider} i jego tłumaczenia. W celu stworzenia polskich odpowiedników obrano podejście polegające na tłumaczeniu maszynowym oraz skryptowym generowaniu różnych wariantów. W dalszej kolejności przeprowadzono przegląd problemu \code{Text-to-SQL} i wybrano 4 modele, na których przeprowadzono dalsze eksperymenty. Ujawniły one wysoką skuteczność rozwiązań \code{RESDSQL} oraz \code{C3}, które następnie zostały zintegrowane z wygodną aplikacją, umożliwiającą generowanie i wykonywanie zapytań. Jej użytkownicy, pomimo zauważenia pewnych problemów, docenili praktyczność opracowanego rozwiązania.

\vspace{1cm}
\noindent{\bf Słowa kluczowe:}
Text-to-SQL, SQL, tłumaczenie, język polski, Spider
\clearpage

% ==================== angielski ==================== 
\section*{Abstract}
\vspace{1cm}

This thesis addresses the currently relevant problem of translating queries from natural language to SQL. It is distinguished by the use of Polish in the role of a natural language, which has not received attention so far. It involved the creation of relevant datasets, experimentation and development of a useful application.

The work began by analyzing existing datasets with particular attention to the \code{Spider} dataset and its translations. The approach of machine translation and scripted generation of various variants was taken to create Polish equivalents. The \code{Text-to-SQL} problem was then reviewed and 4 models were selected on which further experiments were conducted. They revealed the high accuracy of \code{RESDSQL} and \code{C3} solutions, which were then integrated into a convenient application for generating and executing queries. Its users, despite noticing some problems, appreciated the practicality of the developed solution.

\vspace{1cm}
\noindent{\bf Keywords:} 
Text-to-SQL, SQL, translation, Polish language, Spider
\clearpage
