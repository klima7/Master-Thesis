% ==================== polski ==================== 
\section*{Streszczenie}
\vspace{1cm}

Niniejsza praca dyplomowa dotyczy istotnego obecnie problemu tłumaczenia zapytań z języka naturalnego na SQL, lecz z zaznaczeniem bazowania na języku polskim zamiast zwykle spotykanym angielskim. W jej ramach stworzono odpowiednie zbiory danych, przeprowadzono eksperymenty i opracowano użyteczną aplikację.

Działania rozpoczęto od przeanalizowania istniejących zbiorów danych ze szczególnym zwróceniem uwagi na zbiór \code{Spider} i jego tłumaczenia. W celu stworzenia polskich odpowiedników obrano podejście polegające na tłumaczeniu maszynowym oraz skryptowym generowaniu różnych wariantów. W dalszej kolejności przeprowadzono przegląd problemu \code{Text-to-SQL} i wybrano 4 modele, na których przeprowadzono dalsze eksperymenty. Ujawniły one wysoką skuteczność rozwiązań \code{RESDSQL} oraz \code{C3}, które następnie zostały zintegrowane z praktyczną aplikacją umożliwiającą generowanie i wykonywanie zapytań. Jej użytkownicy, pomimo zauważenia pewnych problemów, docenili praktyczne zastosowanie.

\vspace{1cm}
\noindent{\bf Słowa kluczowe:}
Text-to-SQL, SQL, tłumaczenie, polski, Spider
\clearpage

% ==================== angielski ==================== 
\section*{Abstract}
\vspace{1cm}

This thesis deals with the currently relevant problem of translating queries from natural language to SQL, but with an emphasis on basing on Polish instead of the usual English language. It involved creating appropriate datasets, conducting experiments and developing a useful application.

Activities began by analyzing existing datasets with particular attention to the \code{Spider} dataset and its translations. The approach of machine translation and scripted generation of various variants was taken to create Polish equivalents. The \code{Text-to-SQL} problem was then reviewed and 4 models were selected on which further experiments were conducted. They revealed the high accuracy of \code{RESDSQL} and \code{C3} solutions, which were then integrated into an practical application that enabled the generation and execution of queries. Its users, despite noticing some problems, appreciated the practical application.

\vspace{1cm}
\noindent{\bf Keywords:} 
Text-to-SQL, SQL, translation, polish, Spider
\clearpage
