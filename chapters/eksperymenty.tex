\chapter{Eksperymenty}

W niniejszej części przeprowadzone zostaną eksperymenty, polegające na dostosowaniu, wytrenowaniu i przetestowaniu kilku modeli na stworzonych zbiorach. Do tego celu wybrane zostały rozwiązania \code{RAT-SQL}, \code{BRIDE}, \code{RESDSQL} oraz \code{C3}. Każde spośród nich zostanie przeanalizowane w osobnej części. 

Modele \code{RAT-SQL} oraz \code{BRIDGE} to dość stare rozwiązania, które zostały wybrane ze względu na swoją popularność w literaturze oraz wykorzystanie odmiennych podejść. \code{RESDSQL} wyłoniono natomiast, gdyż jest to najwyżej znajdujący się w rankingu \code{Spider} model dedykowany, do którego dostępny jest kod źródłowy. \code{C3} reprezentuje natomiast zupełnie inne podejście, gdyż bazuje na dużym modelu językowym od \code{OpenAI}. Jest to zatem ciekawe rozwiązanie, a koszty wynikające z wykorzystywanym przez nie modelem \code{GPT-3.5-turbo} są całkiem znośne.

W fazie eksperymentów trening modeli postanowiono wykonywać jedynie na zbiorze \code{Spider}. Cztery pozostałe będą wykorzystywane jedynie do przeprowadzania testów. Wynika to z chęci skrócenia czasu treningu oraz umożliwia dokonywanie uczciwych porównań z modelami trenowanymi przez innych badaczy, gdyż niemal zawsze jest to dokonywane właśnie na zbiorze \code{Spider}. Po fazie eksperymentów planuje się wybrać najbardziej obiecujące rozwiązanie i dokonać dłuższego treningu na wszystkich zbiorach.


\section{RAT-SQL}
\code{RAT-SQL} to bardzo rozpoznawalny model, który był krzyżowany na przestrzeni czasu z różnymi innymi algorytmami i jego warianty można znaleźć w rankingu zbioru \code{Spider} na wielu pozycjach. Rozwiązanie to zostało zaproponowane w 2021 roku przez Bailin Wang oraz Richarda Shin \mycite{Wang2019}. Doczekało się dwóch kolejnych iteracji, określanych jako \code{RAT-SQL v2} oraz \code{RAT-SQL v3}. W niniejszej pracy rozważana jest ta ostatnia, której kod dostępny jest na platformie \code{GitHub} w repozytorium Microsoftu \mycite{ratsql-repository}. Model ten produkuje zapytania bez wartości, co ogranicza jego praktyczne zastosowanie.

\subsection{Działanie}
Nazwa modelu \code{RAT-SQL} pochodzi od angielskiego wyrażenia \textit{Relation Aware Transformer}, które doskonale opisuje to, co jest w tym rozwiązaniu najważniejsze. Wykorzystuję ono bowiem enkodowanie oparte na grafie, które zostało zrealizowane przy pomocy sieci typu transformer. Standardowo sieci takie potrafią analizować jedynie sekwencje, więc konieczne było zmodyfikowanie ich w taki sposób, aby działały również dla grafów. Dzięki dokonanym zmianom stały się one świadome relacji pomiędzy poszczególnymi elementami sekwencji.

Wszystkie sieci typu transformer posiadają element odpowiedzialny za uczenie się powiązań pomiędzy elementami wejściowymi w postaci mechanizmu uwagi (ang. attention). Autorzy \mbox{\code{RAT-SQL}} zbadali eksperymentalnie jego działanie dla problemu generowania zapytań SQL i zauważyli, że znajdywane połączenia są jednak dość słabe. Postanowili więc zmodyfikować mechanizm uwagi tak, aby poza znajdywaniem powiązań w standardowy, rozmyty sposób, można było jako dodatkowe wejście podawać powiązania z góry znane. 

Relacje, które postanowiono jawnie zakodować poprzez zmodyfikowany mechanizm uwagi to przede wszystkim przynależność poszczególnych kolumn do tabel i powiązania tworzone przez klucze obce. Wykorzystano również etap \code{schema linking} w celu znalezienia połączeń pomiędzy fragmentami pytania i elementami bazy danych. Zrealizowano to poprzez wydobycie z pytania wszystkich n-gramów o długości od 1 do 5 i wyszukane ich dokładnych lub częściowych dopasowań wśród nazw tabel i kolumn. Połączenia typu \code{value link} znaleziono natomiast poprzez rozważenie każdego słowa z pytania osobno i wyszukiwanie go w każdej z dostępnych kolumn. Wszystkie te relacje tworzą skomplikowany graf, który \code{RAT-SQL} pozwala zakodować.

Przed przekazaniem wszystkich informacji wejściowych do transformera dokonującego enkodowania konieczna jest zamiana wartości tekstowych do postaci wektorowej. Uwaga ta tyczy się nazw tabel i kolumn oraz pytania. \code{RAT-SQL} pozwala dokonać takiej konwersji na dwa sposoby: z wykorzystaniem gotowych embeddingów nauczonych metodą \code{GloVe} lub pretrenowanym modelem \code{BERT}. Po enkodowaniu musi nastąpić dekodowanie, w którym wykorzystano rekurencyjne komórki \code{LSTM} do generowania akcji, pozwalających zbudować drzewo \code{AST}. Gdy to nastąpi to drzewo zamieniane jest na tradycyjną postać zapytania SQL.

\subsection{Modyfikacje dla języka polskiego}
Przystosowanie \code{RAT-SQL} do języka polskiego okazało się dość wymagające ze względu na dużą bazę kodu. W pierwszej kolejności koniecznym było testowe uruchomienie tego rozwiązania, co już sprawiło problemy. W repozytorium autorów dostępny jest plik \code{Dockerfile} teoretycznie umożliwiający proste wystartowanie środowiska, lecz budowa obrazu dockerowego okazała się zwracać błędy ze względu na wygaśnięte klucze \code{GPG}. Po naprawieniu tego problemu niezbędna była zmiana wersji kilku pakietów języka Python ze względu na niezgodności. W ramach tego zmieniono na bardziej aktualną bibliotekę \code{pytorch}, która jest wykorzystywana do tworzenia sieci neuronowych, gdyż starsza wydawała się nie wspierać architektury posiadanej karty graficznej.

Najważniejszą wśród dokonanych modyfikacji była zmiana mechanizmu tworzenia wektorowych reprezentacji tekstów na etapie enkodowania. W oryginalnej implementacji wykorzystany został w tym celu model \code{BERT}, a dokładnie wariant \code{bert-large-uncased-whole-word-masking} z platformy \code{Hugging Face}. Był on trenowany na tekstach angielskich, co go dla wersji polskiej dyskwalifikuje, a ponadto wprowadzana przez niego duża liczba parametrów sprawiała problemy z uruchomieniem na posiadanym sprzęcie. Postanowiono zastąpić go wielojęzycznym odpowiednikiem, trenowanym na 104 językach, oznaczonym na platformie \code{HF} identyfikatorem \code{bert-base-multilingual-uncased}. Posiada on trzykrotnie mniejszą liczbę parametrów, co umożliwia uruchomienie, lecz z pewnością wpływa na niższą skuteczność finalnego rozwiązania. Dostępne są również warianty trenowane tylko na języku polskim, lecz je wykluczono, ponieważ jak wcześniej zauważono, nazwy tabel i kolumn nawet w polskich bazach są najczęściej utrzymywane po angielsku -- całkowicie polski wariant modelu \code{BERT} nie pokryłby tego przypadku. 

Jako alternatywną metodę tworzenia reprezentacji tekstów \code{RAT-SQL} wykorzystuję embeddingi nauczone metodą \code{GloVe}. Nie udało się niestety znaleźć dla nich odpowiednika wielojęzycznego, więc wykorzystano embeddingi całkowicie polskie, zamieszczone w repozytorium \code{polish-nlp-resources} \mycite{polish-nlp-resources}. Dokładnie wybrano wariant \code{300d}, w którym są one wektorami o długości trzystu elementów, ponieważ takiej długości były również oryginalnie zastosowanie embeddingi angielskie.

Wśród innych dokonanych zmian znalazła się modyfikacja listy słów nazywanych \code{stop words}, czyli powszechnie występujących, lecz nie niosących w sobie wiele informacji (i, w, z, na, do, się, o). Słowa takie są pomijane podczas poszukiwania częściowych dopasowań na etapie \code{schema linking} i należało podmienić je na polskie odpowiedniki. Ponadto w rozważanej bazie kodu kilka fragmentów wykorzystuję lematyzację, czyli zamianę słów na ich bazowe formy. W oryginalnej postaci zakłada ona pracę na tekstach angielskich, więc należało dostarczyć polską implementację, do czego wykorzystano bibliotekę \code{Stanza}. Ostatecznie etap \code{schema linking} bazował na znajdywaniu częściowych powtórzeń pomiędzy elementami bazy a fragmentami pytania. Jako że nazwy tabel i kolumn często nie zawierają polskich znaków, a pytania je posiadają, to słusznym działaniem wydało się zmodyfikowanie mechanizmu porównywania tak, aby ignorował różnicę w znakach specjalnych.

\subsection{Eksperymenty}

W ramach eksperymentów postanowiono wytrenować i przetestować model \code{RAT-SQL} w obu wariantach: z enkodowaniem tekstów za pomocą pretrenowanego modelu \code{BERT} oraz z enkodowaniem za pomocą embeddingów \code{GloVe}. W pierwszej konfiguracji model liczy 244 miliony parametrów, a w drugiej zaledwie 17 milionów, z czego wszystkie podlegają optymalizacji. Dla wariantu z \code{BERT} zdecydowano się wytrenować dwie wersje, gdzie jedna zostanie nauczona na polskim zbiorze \code{Pol-Spider}, a druga na zbiorze \code{mix-spider}, który zawiera dodatkowo oryginalny zbiór angielski. Niektóre artykuły podają bowiem, że modele trenowane na tego typu połączonych zbiorach tłumaczonych i angielskich osiągają lepsze wyniki niż trenowane jedynie na tłumaczeniach \mycite{Bakshandaeva2022,SpiderPortuguese}, co postanowiono zweryfikować. Po wariancie z \code{GloVe} spodziewano się, że osiągnie słabsze wyniki od \code{BERT}, gdyż zgodnie z artykułem wprowadzającym \code{RAT-SQL} tak właśnie powinno się stać. Jak zostało wcześniej wspomniane, wykorzystane embeddingi \code{GloVe} trenowane były jedynie na tekstach polskich. Z tego powodu ten wariant postanowiono nauczyć dla bardziej ograniczonego scenariusza, w którym nazwy tabel i kolumn zawsze będą w języku polskim i trening przeprowadzono na zbiorze \code{Pol-Spider-PL}.

Autorzy \code{RAT-SQL} oryginalnie trenowali wariant \code{BERT} przez 80 000 kroków, co postanowiono skrócić o połowę. Pomimo tego równoległy trening na obu wspomnianych zbiorach, na dwóch posiadanych kartach graficznych, zajął prawie trzy doby. Wariant \code{GloVe} trenowany był natomiast oryginalnie przez 40 000 kroków, co postanowiono pozostawić bez zmian i w tym przypadku zajęło to niecałe dwie doby. 

\begin{figure}[ht]
  \begin{center}
    \begin{tikzpicture}
      \begin{axis}[
        width=\linewidth,
        height=\fpeval{0.5*\linewidth},
        xmin=0, xmax=40700,
        grid=major,
        xlabel=krok treningu,
        ylabel={EM Without Values [\%]},
        legend cell align={left},
        legend pos=south east,
      ]
        \addplot table[x=step,y expr=\thisrow{em} * 100,col sep=comma] {plots/ratsql_training_bert.csv}; 
        \addplot table[x=step,y expr=\thisrow{em} * 100,col sep=comma] {plots/ratsql_training_bert_mix.csv}; 
        \addplot table[x=step,y expr=\thisrow{em} * 100,col sep=comma] {plots/ratsql_training_glove.csv}; 
        \legend{BERT + Pol-Spider, BERT + mix-spider, GloVe}
      \end{axis}
    \end{tikzpicture}
    \lcaption{Wyniki modeli na zbiorze testowym w trakcie trwania treningu}{Warianty \code{BERT} testowano na zbiorze \code{Pol-Spider}, natomiast \code{GloVe} na zbiorze \code{Pol-Spider-PL}.}
    \label{plot:ratsql-accuracy}
  \end{center}
\end{figure}

Podczas treningu wagi modeli były regularnie zapisywane na dysku w odstępie 1000 kroków. Po zakończeniu każdy taki zestaw wag został przetestowany z wykorzystaniem metryki \code{EM Without Values}, co zajęło sumarycznie kilkanaście godzin obliczeń. Warianty z \code{BERT} testowane były na zbiorze \code{Pol-Spider}, natomiast wariant z \code{GloVe} na zbiorze \code{Pol-Spider-PL}. Wykres przedstawiający zmianę skuteczności modeli wraz z kolejnymi krokami treningu przedstawiono na rysunku \ref{plot:ratsql-accuracy}. Ostatecznie z każdego z trzech zestawów wag wybrano do dalszych testów ten, który podczas przedstawionej ewaluacji otrzymał najlepsze rezultaty. 

\subsection{Wyniki}
Trzy wytrenowane modele przetestowano na różnych konfiguracjach zbioru \code{Spider}. Wyniki tej analizy przedstawiono w tabeli \ref{tab:ratsql-results}. Można z niej odczytać, że w każdym przypadku najlepiej nauczył się wariant \code{BERT} trenowany na zbiorze \code{mix-spider}, nieco niższą skuteczność osiągnął \code{BERT} uczony na zbiorze \code{Pol-Spider}, natomiast model wykorzystujący embeddingi \code{GloVe} znacznie od tych dwóch odstaję. Z wykorzystaniem najlepszego modelu przeprowadzono dalsze testy, w których dokonano jego ewaluacji na wszystkich zbiorach pokrewnych z podziałem na poziomy trudności zapytań. Wyniki tych testów przedstawiono w tabeli \ref{tab:ratsql-difficulty}.

\begin{table}[ht]
    \centering
    \begin{tabular}{|l|l|r|r|r|r|r|}
        \hline
        \multirow{2}{*}{\thead{\small{Wariant}}} &
        \multirow{2}{*}{\small{\thead{zbiór\\treningowy}}} &
        \multicolumn{4}{c|}{\thead{\small{Zbiór testowy}}} \\
        \cline{3-6}
        \multirow{2}{*}{} &
        \multirow{2}{*}{} &
        \thead{\small{Pol-Spider}} &
        \thead{\small{Pol-Spider-PL}} &
        \thead{\small{Pol-Spider-EN}} &
        \thead{\small{En-Spider}} \\
        \hline
        BERT & Pol-Spider & 53,1 & 56,6 & 49,7 & 42,8 \\
        BERT & mix-spider & 58,1 & 61,0 & 55,1 & 68,8 \\
        GloVe & Pol-Spider-PL & 19,3 & 27,5 & 11,2 & \s2,8 \\
        \hline
    \end{tabular}
    \lcaption{Wyniki modeli otrzymane metodą \code{RAT-SQL} na wariantach zbioru \code{Spider}}{Wartości metryk podano w procentach.}
    \label{tab:ratsql-results}
\end{table}

\begin{table}[ht!]
    \centering
    \begin{tabular}{|l|R{0.10\textwidth}|R{0.10\textwidth}|R{0.10\textwidth}|R{0.10\textwidth}|R{0.10\textwidth}|}
        \hline
        \thead{Zbiór} & \thead{Easy} & \thead{Medium} & \thead{Hard} & \thead{Extra\\Hard} & \thead{Razem} \\
        \hline
        Pol-Spider & 73,8 & 58,7 & 50,9 & 40,4 & 58,1 \\
        Pol-Spidersyn & 58,5 & 45,1  & 41,2  & 26,9  & 44,5 \\
        Pol-Spiderdk & 60,9 & 36,4 & 31,1 & 17,6 & 37,0 \\
        Pol-Sparc & 62,1 & 42,2  & 10,0  & 25,0  & 53,6 \\
        Pol-Cosql & 59,6 & 47.5  & 22,1  & 29,4  & 48,7 \\
        \hline
    \end{tabular}
    \lcaption{Wyniki najlepszego modelu \code{RAT-SQL} na zbiorach pokrewnych}{Najlepszym modelem jest wariant \code{BERT} uczony na zbiorze dwujęzycznym. Wartości metryk podano w procentach.}
    \label{tab:ratsql-difficulty}
\end{table}

\subsection{Analiza}
Istotną obserwacją jest to, że model trenowany na połączeniu zbioru polskiego i angielskiego nauczył się lepiej przewidywać zapytania SQL od tego uczonego wyłącznie na zbiorze polskim. Jest to zgodne z eksperymentami przeprowadzonymi i opisanymi również przez innych badaczy, które podsumowano w tabeli \ref{tab:ratsql-translations-results}. Można z niej odczytać rezultaty, które udało się osiągnąć twórcom dwóch innych tłumaczeń zbioru \code{Spider}, którzy także eksperymentowali z metodą \code{RAT-SQL} i nauką na różnych wariantach swoich zbiorów. W przypadku zbioru portugalskiego zysk otrzymany na poszerzonym zbiorze jest dość niewielki, natomiast dla zbioru rosyjskiego wynosi aż 6 punktów procentowych. Korzyść dla tłumaczenia polskiego jest dość duża i bliższa temu ostatniemu. 

\begin{table}[ht!]
    \centering
    \begin{tabular}{|c|r|r|}
        \hline
        \multirow{2}{*}[-0.8em]{\thead{Tłumaczenie \\zbioru Spider}} & \multicolumn{2}{c|}{\thead{Zbiór treningowy}} \\
        \cline{2-3}
        \multirow{2}{*}{} & \thead{Tłumaczenie} & \thead{Tłumaczenie\\ + angielski} \\
        \hline
        \multicolumn{1}{|l|}{Rosyjskie (\code{PAUQ})} & 51,0 & 57,0 \\
        \multicolumn{1}{|l|}{Portugalskie} & 58,8 & 59,5 \\
        \multicolumn{1}{|l|}{Polskie (\code{Pol-Spider})} & 53,1 & 58,1 \\
        \hline
    \end{tabular}
    \lcaption{Wyniki osiągane przez \code{RAT-SQL} dla różnych zbiorów treningowych}{Dane dotyczą modelu w wariancie \code{BERT}. Wartości w komórkach tabeli to wyniki metryki \code{EM Without Values} wyrażone procentowo.}
    \label{tab:ratsql-translations-results}
\end{table}

Wyniki osiągnięte przez model w wariancie z \code{GloVe} okazały się być znacznie niższe w porównaniu z wariantami \code{BERT}. Było to spodziewane, biorąc pod uwagę choćby liczbę parametrów, jednak wydaje się, że wyniki metryk mogłyby być nieco wyższe. Wytrenowany przez twórców \code{RAT-SQL} model odstaję od wariantu \code{BERT} o około 7 punktów procentowych, a w naszym przypadku różnica wynosi ponad 20 punktów. Powodem tego może być przypuszczalnie gorsza jakość embeddingów \code{GloVe} dla języka polskiego niż angielskiego, lecz dalszych eksperymentów zaniechano. Model w tym wariancie mógłby stanowić nieco słabszą alternatywę, którą można szybko wytrenować i uruchamiać na mniej wydajnych urządzeniach. Aktualna dokładność wydaję się jednak za niska, aby to znalazło praktyczne zastosowanie.

Dane liczbowe w tabeli \ref{tab:ratsql-difficulty} pokazują, że wraz ze zwiększaniem się poziomu trudności zapytań maleje skuteczność modelu, co jest spodziewanym zachowaniem. Niewielkie rozbieżności od tej reguły można zaobserwować jedynie dla zbiorów \code{Pol-Cosql} oraz \code{Pol-Sparc}. Przyczyną tego jest zapewne niewielka liczba znajdujących się w nich trudnych zapytań, przez co kilka pozornie wymagających mogło znacząco zachwiać wynikami.

Efekty uzyskane na zbiorze \code{Pol-Spidersyn} okazały się wyraźnie słabsze w porównaniu do tych na \code{Pol-Spider}, bo różnica wynosi aż 13,57 punktów procentowych. Potwierdza to obserwację przedstawioną w artykule wprowadzającym \code{Spider-Syn}. Mówi ona, że modele mają duże problemy z odpowiadaniem na pytania, w których użytkownik nie znając dokładnie schematu bazy danych, posługuję się synonimami.

Niskie wyniki, chociaż również spodziewane, osiągnął model na zbiorze \code{Spider-DK}. Zawiera on bowiem pytania wymagające znajomości wiedzy domenowej, z którą uważa się, że modele sobie często nie radzą. Spadek, który nastąpił w tym przypadku względem zbioru \code{Pol-Spider}, wynosi 21,07 punktów procentowych. W celu dokonania głębszej analizy możliwe jest obliczenie dokładności modelu w obrębie każdego typu wiedzy domenowej z osobna, czego jednak postanowiono nie robić.

\section{Model BRIDGE}
\code{BRIDGE} to rozwiązanie, które zostało opublikowane w roku 2020 przez Xi Victorię Lin, Richarda Sochera oraz Caiming Xionga \mycite{Lin2020}. Powstało więc w podobnym czasie jak \code{RAT-SQL}, ale w odróżnieniu od niego generuje kompletne zapytania SQL z wartościami, co jest istotną zaletą. Jego kod źródłowy jest umieszczony na platformie \code{GitHub} \mycite{bridge-repository}.

\subsection{Działanie}
Działanie \code{BRIDGE} znacząco odbiega od \code{RAT-SQL}. W tym przypadku wykorzystano enkodowanie oparte na serializacji informacji wejściowych do długiego tekstu i przekazywaniu go do pretrenowanego modelu \code{BERT}. Dekodowanie opiera się natomiast na generowaniu słów jedno po drugim, zamiast generowania akcji tworzących drzewo \code{AST}. Na obu etapach zastosowano ciekawe techniki, które dodatkowo poprawiają skuteczność.

Enkodowanie opiera się na skonstruowaniu sekwencji tekstowej zawierającej wszystkie nazwy tabel, gdzie po każdej nazwie tabeli znajduje się lista zawartych w niej kolumn. Na początku tej sekwencji doklejane jest dodatkowo rozpatrywane pytanie. Wcześniej wykonywany jest etap \code{schema linking} w celu znalezienia połączeń typu \code{value link} i odnalezione dopasowania w zawartości bazy danych są  wstawiane po nazwie odpowiedniej kolumny. W celu zachowania znaczenia poszczególnych części stworzonej sekwencji wykorzystywane są specjalne tokeny \code{[T]}, \code{[C]} oraz \code{[V]}, które są wstawiane odpowiednio przed każdą nazwą tabeli, kolumny i wartością. Stworzony tekst jest następnie przetwarzany przez model językowy \code{BERT} i dodane po nim dwie lekkie warstwy \code{LSTM}. W ten sposób uzyskiwane są wykorzystywane przez dekoder wektorowe reprezentacje wszystkich tabel, kolumn i pytania. Reprezentacje kolumn są jednak dodatkowo wzbogacane za pomocą trenowanych równolegle wektorów reprezentujących metainformacje, takie jak bycie kluczem podstawowym, bycie kluczem obcym, czy posiadanie konkretnego typu danych. Łącznie podstawowych reprezentacji z tymi metainformacjami dokonuje się prostą warstwą liniową.

Wykorzystany dekoder generuje wyjściowe zapytanie słowo po słowie. Nie jest to jednak typowy dekoder, jak te wykorzystywane w modelach językowych, ponieważ na każdym kroku dekodowania, poza wyprodukowaniem jednego tokena ze słownika, może dokonać także kopiowania go z pytania lub spośród nazw tabel i kolumn. Aby umożliwić takie zachowanie standardowy dekoder, bazujący na \code{LSTM}, został połączony z siecią typu \code{Pointer Network} \mycite{Vinyals2015}, która potrafi wskazywać konkretne pozycje w sekwencjach wejściowych. Poza tym dekoder został nauczony w taki sposób, aby produkować zapytania w kolejności wykonywania, czyli takiej, w której wykonałby je silnik bazodanowy. Powoduje to, że nazwy tabel generowane są przed nazwami kolumn i dzięki temu podczas generowania nazwy kolumny można ograniczyć się jedynie do kolumn dostępnych we wcześniej wymienionych tabelach.

\subsection{Modyfikacje dla języka polskiego}
Przystosowanie \code{BRIDGE} do języka polskiego okazało się znacznie prostsze niż to było w przypadku \code{RAT-SQL}. Tutaj również pracę rozpoczęto od uruchomienia oryginalnego rozwiązania, więc napisano plik \code{Dockerfile}, który tworzy obraz dockerowy zawierający kompletne środowisko. Jedynym problemem okazał się brak wśród zależności projektu biblioteki \code{numpy}, która była wykorzystywana. Naprawiono to poprzez dodanie odpowiedniej linii kodu.

Jedyną ważną modyfikacją dla języka polskiego okazała się zmiana wykorzystywanego modelu \code{BERT} z wariantu \code{bert-large-uncased} na \code{bert-base-multilingual-uncased}, oba dostępne są na platformie \code{HF}. Jest to działanie analogiczne do zmiany zaaplikowanej w \code{RAT-SQL}. Uzasadnienie również jest podobne: zmieniono model na wielojęzyczny, by poradził sobie z językiem polskim oraz na mniejszy, by umożliwić naukę na posiadanym sprzęcie. Tak zmodyfikowany model \code{BRIDGE} liczy sobie 174 miliony parametrów i wszystkie podlegają treningowi.

\subsection{Eksperymenty}
W ramach eksperymentu chciano dokonać treningu modelu \code{BRIDGE} i go przetestować. Nauki dokonano na zbiorze \code{Mix-Spider}, ponieważ wcześniej przeprowadzone eksperymenty z \code{RAT-SQL} pokazały, że model trenowany na połączonym zbiorze polskim i angielskim nauczył się lepiej. Oczywiście nie można założyć na tej podstawie, że każdy model uczony na połączonym zbiorze będzie skuteczniejszy. Można jednak przypuszczać, że zwykle tak jest.

\begin{figure}[ht!]
  \begin{center}
    \begin{tikzpicture}
      \begin{axis}[
        width=\linewidth,
        height=\fpeval{0.5*\linewidth},
        xmin=0, xmax=37,
        grid=major,
        xlabel={Czas treningu [godziny]},
        ylabel={EM Without Values [\%]},
      ]
        \addplot table[x=time,y expr=\thisrow{dev_exact_match} * 100,col sep=comma] {plots/bridge_training.csv}; 
      \end{axis}
    \end{tikzpicture}
    \lcaption{Wyniki modelu \code{BRIDGE} w czasie trwania treningu}{Model sprawdzano na części testowej zbioru \mbox{\code{Pol-Spider}}.}
    \label{plot:bridge-accuracy}
  \end{center}
\end{figure}

Model uczono przez 36 godzin na jednostce wyposażonej w kartę \code{Nvidia RTX 3080}, a zaimplementowany wewnątrz procedury treningu kod dokonywał w tym czasie regularnego obliczania metryki \code{EM Without Values} na części testowej zbioru \code{Pol-Spider}. Wykres przedstawiający zmianę wartości tej metryki na przestrzeni czasu przedstawiono na rysunku \ref{plot:bridge-accuracy}. Na wykresie widać, że skuteczność modelu rosła wyraźnie przez połowę czasu treningu, a w drugiej połowie już nie widać tendencji wzrostowej -- dlatego naukę postanowiono przerwać. Jako finalny został wybrany model z punktu, w którym wartość wspomnianej metryki była najwyższa.

\subsection{Wyniki}
Wyniki przeprowadzonego eksperymentu przedstawiono w tabeli \ref{tab:bridge-difficulty}. Jest ona znacznie bardziej rozbudowana niż to było w przypadku rozwiązania \code{RAT-SQL}, ponieważ ten model zwraca zapytania z wartościami. Możliwe więc stało się obliczenie metryk \code{EM} oraz \code{EX}.

\begin{table}[H]
    \centering
    \begin{tabular}{|l|r|r|r|r|r|}
        \hline
        \thead{Zbiór} & \thead{Easy} & \thead{Medium} & \thead{Hard} & \thead{Extra Hard} & \thead{Razem} \\
        \hline
        Pol-Spider & 
        \threevals{79,4}{71,2}{82,9} &
        \threevals{59,8}{51,9}{68,9} &
        \threevals{43,1}{40,2}{63,8} &
        \threevals{34,6}{31,9}{54,5} &
        \threevals{57,6}{51,4}{69,1} \\
        
        Pol-Spider-PL &
        \threevals{80,6}{72,6}{84,3} &
        \threevals{60,3}{52,5}{70,0} &
        \threevals{44,8}{40,2}{64,9} &
        \threevals{34,9}{31,3}{53,6} &
        \threevals{58,5}{51,8}{69,9} \\
        
        Pol-Spider-EN &
        \threevals{78,2}{69,8}{81,5} &
        \threevals{59,2}{51,3}{67,9} &
        \threevals{41,4}{40,2}{62,6} &
        \threevals{34,3}{32,5}{55,4} &
        \threevals{56,8}{50,9}{68,3} \\
        
        En-Spider &
        \threevals{86,7}{82,3}{85,1} &
        \threevals{67,9}{62,8}{71,7} &
        \threevals{51,7}{48,3}{58,0} &
        \threevals{40,4}{38,0}{43,4} &
        \threevals{65,3}{61,0}{68,1} \\
        
        \hline
        
        Pol-Spidersyn &
        \threevals{63,2}{56,1}{71,6} &
        \threevals{48,3}{41,7}{59,3} &
        \threevals{37,1}{36,4}{52,6} &
        \threevals{23,1}{21,5}{43,8} &
        \threevals{45,7}{40,8}{58,4} \\
        
        Pol-Spiderdk &
        \threevals{58,6}{51,4}{63,6} &
        \threevals{39,4}{32,9}{49,4} &
        \threevals{22,3}{22,3}{44,6} &
        \threevals{15,7}{13,8}{31,9} &
        \threevals{36,4}{31,5}{48,2} \\
        
        Pol-Sparc &
        \threevals{61,7}{60,0}{68,1} &
        \threevals{38,8}{44,1}{60,2} &
        \threevals{\s3,3}{25,0}{43,3} &
        \threevals{25,0}{17,6}{43,8} &
        \threevals{52,0}{47,6}{64,3} \\
        
        Pol-Cosql &
        \threevals{60,0}{52,5}{69,2} &
        \threevals{44,1}{34,7}{63,6} &
        \threevals{25,0}{19,1}{54,4} &
        \threevals{17,6}{14,7}{47,1} &
        \threevals{47,6}{40,2}{63,9} \\
        
        \hline
    \end{tabular}
    \lcaption{Wyniki modelu \code{BRIDGE} na poszczególnych zbiorach}{Wartości w każdej komórce posiadają format \mbox{EM Without Values / EM / EX}.}
    \label{tab:bridge-difficulty}
\end{table}

\subsection{Analiza}
Zestawiając ze sobą wyniki osiągnięte metodą \code{BRIDGE} oraz \code{RAT-SQL} trudno jest wskazać pod względem metryki \code{EM Without Values}, która jest jedyną wspólną, model lepszy. Na częściach testowych wszystkich wariantów zbioru \code{Spider} analizowany teraz model \code{BRIDGE} okazuję się działać lepiej. Na zbiorach pokrewnych \code{Pol-Spiderdk}, \code{Pol-Sparc} oraz \code{Pol-Cosql} jednak to \code{RAT-SQL} jest skuteczniejszy. Można z tego wyciągnąć wniosek, że \code{BRIDGE} sprawdza się lepiej, jeżeli trenowany i testowany jest na podobnych danych. \code{RAT-SQL} wydaje się natomiast posiadać bardziej rozwiniętą umiejętność generalizacji i wykorzystania wiedzy domenowej, przez co osiąga lepsze wyniki na zbiorach wyraźnie odbiegających od treningowego. 

Metoda \code{BRIDGE} nie jest już tak popularna w literaturze jak \code{RAT-SQL} i eksperymenty z jej wykorzystaniem przeprowadzili jedynie autorzy rosyjskiego tłumaczenia zbioru \code{Spider}, czyli autorzy \code{PAUQ}. Zestawienie osiągniętych przez nich wyników z niniejszymi przedstawiono w tabeli \ref{tab:bridge-translations-results}. Widać, że zgodnie z metryką \code{EM Without Values} na zbiorze polskim udało się osiągnąć rezultaty nieco lepsze. Możliwą przyczyną jest przypuszczalnie większa liczba danych polskich, na których wielojęzyczny model \code{BERT} mógł być trenowany, albo też wyższy stopień skomplikowania języka rosyjskiego. Różnica w wynikach metryki \code{EX} jest natomiast bardzo duża, na korzyść zbioru polskiego. Przyczyną tego jest pewne zakłamanie metryki \code{EX} dla języka polskiego, o czym wcześniej wspomniano. W pełni uzasadnione jej zaaplikowanie wymagałoby bowiem przetłumaczenia zawartości wszystkich baz danych, czego dla polskiego zbioru nie zrobiono, a w przypadku \code{PAUQ} zostało dokonane częściowo. Szczególnie więc w tym przypadku, ze względu tłumaczenia zawartości baz w tylko jednym zbiorze, należy uważać z porównywaniem tych metryk ze sobą. Mimo wszystko przedstawione zestawienie stanowi potwierdzenie tego, że modyfikacji metody \code{BRIDGE} dla języka polskiego dokonano poprawnie.

\begin{table}[ht!]
    \centering
    \begin{tabular}{|c|r|r|}
        \hline
        \multirow{2}{*}[-0.8em]{\thead{Tłumaczenie \\zbioru Spider}} & \multicolumn{2}{c|}{\thead{Zbiór treningowy}} \\
        \cline{2-3}
        \multirow{2}{*}{} & \thead{Tłumaczenie} & \thead{Tłumaczenie\\ + angielski} \\
        \hline
        \multicolumn{1}{|l|}{Rosyjskie (\code{PAUQ})} & \twovals{52,0}{48,0} & \twovals{55,0}{50,0} \\
        \multicolumn{1}{|l|}{Polskie (\code{Pol-Spider})} & \twovals{\varendash[20pt]}{\varendash[20pt]} & \twovals{57,6}{69,1} \\
        \hline
    \end{tabular}
    \lcaption{Zestawienie wyników modelu \code{BRIDGE} dla polskiego i rosyjskiego tłumaczenia}{Komórki tabeli zawierają wyniki metryk w formacie EM Without Values / EX.}
    \label{tab:bridge-translations-results}
\end{table}

Analizując dalej tabelę \ref{tab:bridge-difficulty}, można zauważyć, że wyniki metryki \code{EM Without Values} są zawsze większe od \code{EM}, co z jednej strony jest oczywiste, ale warto to podkreślić. Różnica pomiędzy nimi wynosi średnio 5,6 punktów procentowych, co oznacza, że ponad pięć procent generowanych zapytań jest poprawnych pod względem struktury, lecz posiada błąd w przewidzianych wartościach. Widać więc, że generowanie wartości stanowi istotne wyzwanie. Dla \code{RESDSQL} aktualne są spostrzeżenia poczynione dla poprzedniego modelu, takie jak spadek skuteczności wraz ze wzrostem poziomu trudności, czy niższa dokładność w przypadku zbiorów pokrewnych.



\section{RESDSQL}
\code{RESDSQL} to dość nowe rozwiązanie, gdyż powstało na początku 2023 roku. Zostało opublikowane w artykule \bibtitle{RESDSQL: Decoupling Schema Linking and Skeleton Parsing for Text-to-SQL} \mycite{Li2023resdsql} przez Haoyang Li i innych. Na chwilę pisania niniejszej pracy jest to najwyżej znajdujące się w rankingu \code{Spider} rozwiązanie, które nie opiera się na wykorzystaniu dużych pretrenowanych modeli językowych od \code{OpenAI} i do którego dostępny jest kod źródłowy \mycite{resdsql-repository}.

\subsection{Działanie}
\code{RESDSQL} bazuje w dużej mierzę na pretrenownym modelu językowym \code{T5} \mycite{Raffel2019}. Jest to kompletny model typu transformer, który służy do generowania tekstowych odpowiedzi na podstawie tekstowych informacji wejściowych i został już wcześniej wytrenowany na obszernym zbiorze, więc zawiera dużą ilość wiedzy. Występuje  \code{RESDSQL} dokonuje dotrenowania tego modelu tak, aby zwracał zapytania SQL będące odpowiedzią na podawane pytania. Podczas konstruowania sekwencji wejściowej oraz dekodowania sekwencji wyjściowej dodaje jednak ciekawe techniki, które poprawiają jego skuteczność dla rozważanego problemu \code{Text-to-SQL}. Ich głównym celem jest rozdzielenie od siebie generowania szkieletu zapytań i uzupełniania go konkretnymi nazwami tabel i kolumn, co z resztą zostało podkreślone w tytule artykułu.

Ważnym elementem, odróżniającym \code{RESDSQL} od wcześniej opisanych rozwiązań jest to, że do enkodera zawartego wewnątrz \code{T5} nie są przekazywane wszystkie nazwy tabel i kolumn, lecz tylko te, które zostały uznane w kontekście danego pytania za najbardziej istotne. Celem tego działania jest odciążenie enkodera z wykonywania skomplikowanego procesu \code{schema linking}. Aby dostać potrzebną informację o istotności poszczególnych tabel i kolumn dla podanego pytania trenowana jest wcześniej całkowicie niezależna sieć, która przyjmuję pytanie oraz wszystkie nazwy tabel i kolumn i dla każdej z nich zwraca prawdopodobieństwo, które może być interpretowane jako stopień istotności. Sieć ta określana jest przez swoich autorów mianem \code{cross-encoder}. Ostatecznie do modelu \code{T5} przekazywane są 4 najistotniejsze tabele, a dla każdej z nich 5 najważniejszych kolumn.

Z punktu widzenia dekodowania najważniejszą nowością jest to, że model \code{T5} nie jest dotrenowywany tak, aby od razu produkować kompletne zapytania, lecz na początku zwrócić jego szkielet, a dopiero za nim wykonywalne zapytanie. Uzasadnieniem opisanego zachowania jest to, że początkowe stworzenie szkieletu jest w miarę prostym problemem, więc powinno mieć dużą dokładność. Dekodery w modelach językowych posiadają własność nazywaną autoregresyjnością, która polega na tym, że podczas generowania kolejnych fragmentów odpowiedzi brany jest pod uwagę tekst wygenerowany do tej pory. Dzięki temu model \code{T5} podczas produkowania drugiem części odpowiedzi, która zawiera kompletne zapytanie SQL, może odwoływać się do wcześniejszej części z szablonem i traktując ją jako pewnego rodzaju notatnik uzupełnić brakujące elementy.

Ciekawą techniką z którą \code{RESDSQL} można łączyć jest \code{NatSQL} \mycite{Gan2021natsql}. Wprowadza ona alternatywną reprezentację dla zapytań SQL, która bardziej przypomina język naturalny i w związku z tym jest łatwiejsza do nauczenia i generowania przez większość modeli. Jednocześnie jest ona na tyle jednoznaczna, że można przekonwertować ją na tradycyjne zapytania SQL. Zgodnie z \mycite{Li2023resdsql} zastąpienie tradycyjnych zapytań za pomocą \code{NatSQL} pozwoliło na zbiorze \code{Spider} zwiększyć skuteczności mierzonej metryką \code{EM} o około 2 procent.

\subsection{Modyfikacje dla języka polskiego}
Rozwiązanie \code{RESDSQL} okazało się wyjątkowo proste do przystosowania dla języka polskiego, bo nie trzeba było wykonywać żadnych kreatywnych modyfikacji. Metoda ta została bowiem już wcześniej wykorzystana do pracy na rosyjskim zbiorze \code{PAUQ} i stosowny kod znajdywał się w repozytorium. Wystarczyło dodać kilka nowych skryptów, które nieznacznie różnią się od istniejących.

Najbardziej istotną modyfikacją, niezbędną do nauki na polskim, czy też rosyjskim tłumaczeniu jest zmiana wykorzystywanego modelu \code{T5} na \code{mT5} \mycite{Xue2020}. Różnica pomiędzy nimi jest jedynie taka, że pierwszy został nauczony na tekstach angielskich, natomiast drugi na zbiorze zawierającym 101 różnych języków, w tym wspomniane dwa.

Okazuję się, że niestety dla języka polskiego nie można łatwo wykorzystać obiecującego połączenia niniejszej metody z \code{NatSQL}. Przyczyną tego jest brak udostępnienia przez autorów \code{NatSQL} skryptu przekształcającego tradycyjne zapytania SQL do zaprojektowanej przez nich postaci. Opublikowane są jedynie zapytania ze zbioru \code{Spider} po dokonaniu takiej konwersji. W przypadku rosyjskiego tłumaczenia udało się zastosować \code{NatSQL}, lecz wymagało to przekształcenia każdego zapytania w sposób ręczny. Dla zbioru polskiego należałoby postąpić analogicznie, co wymagałoby wiele żmudnej pracy, której postanowiono uniknąć.

\subsection{Eksperymenty}
W ramach eksperymentu postanowiono dokonać nauki \code{RESDSQL} na zbiorze \code{mix-spider}, ponieważ jak wcześniej zauważono, nauka na dwujęzycznych zbiorach wydaje się mieć lepsze efekty niż na zbiorach jednojęzycznych. Konieczny był wybór konkretnego wariantu dotrenowywanego modelu \code{mT5}, ponieważ występuje on w kilku rozmiarach. Na mocniejszej z dwóch posiadanych kart graficznych, czyli \code{RTX 3080}, udało się uruchomić jedynie wariant najmniejszy - \code{mt5-base}. Sprawdzając poszczególne rozmiary stosowano minimalną wielkość wsadu (ang. batch size), czyli ilość danych jednocześnie przekazywanych do modelu, aby zmniejszyć zapotrzebowanie na pamięć VRAM. Po dokonaniu głębszej ingerencji w procedurę treningu udało się ostatecznie uruchomić ją także dla większego modelu. Wymagało to jednak załadowania wag i treningu z wykorzystaniem arytmetyki 16-bitowej zamiast powszechnie stosowanej 32-bitowej. Może się to wiązać z pojawieniem podczas nauki różnych niestabilności, więc z powodu braku doświadczenia z tego typu treningiem postanowiono pozostać przy 

Zgodnie z wcześniejszym opisem wykorzystanie \code{RESDSQL} wymaga w pierwszej kolejności wytrenowania sieci \code{cross-encoder}, która dokonuje oceny istotności tabel i kolumn. Trening ten wykonywano przez 4 godziny, a wykres przedstawiający zmianę skuteczności modelu w jego trakcie na części testowej umieszczono na rysunku \ref{plot:resdsql-classifier-accuracy}. Wspomniana skuteczność jest tutaj wyrażana za pomocą metryki \code{AUC} \mycite{Ling2003}, popularnej dla zadań klasyfikacji. Po 4 godzinach trening został automatycznie przerwany, ponieważ od dłuższego czasu nie nastąpiło zwiększenie skuteczności rozumianej jako sumy \code{AUC} dla tabel i kolumn. Jako produkt tego etapu został wybrany model z zestawem wag, dla którego metryka ta była najwyższa.

\begin{figure}[ht!]
\centering
\begin{subfigure}{0.49\textwidth}
    \begin{tikzpicture}
      \begin{axis}[
        width=\linewidth,
        height=\linewidth,
        xmin=0, xmax=4.1,
        grid=major,
        xlabel={Czas treningu [godziny]},
        ylabel=AUC,
        legend cell align={left},
        legend pos=south east,
      ]
        \addplot table[x=time,y=table,col sep=comma] {plots/resdsql_classifier_training.csv}; 
        \addplot table[x=time,y=column,col sep=comma] {plots/resdsql_classifier_training.csv}; 
        \legend{dla tabel, dla kolumn}
      \end{axis}
    \end{tikzpicture}
    \caption{model oceniający istotność tabel i kolumn}
    \label{plot:resdsql-classifier-accuracy}
\end{subfigure}
\hfill
\begin{subfigure}{0.49\textwidth}
    \begin{tikzpicture}
      \begin{axis}[
        width=\linewidth,
        height=\linewidth,
        xmin=0, xmax=14,
        grid=major,
        xlabel={Czas treningu [godziny]},
        ylabel=EM without values,
      ]
        \addplot table[x=step,y=em,col sep=comma] {plots/resdsql_t5_training.csv}; 
      \end{axis}
    \end{tikzpicture}
    \caption{model przewidujący zapytania}
    \label{plot:resdsql-t5-accuracy}
\end{subfigure}
\caption{Wykresy skuteczności modeli \code{RESDSQL} na części testowej zbioru \code{pol-spider} wraz z kolejnymi krokami treningu}
\label{plot:resdsql-accuracy}
\end{figure}

Drugim etapem nauki było dotrenowywanie modelu \code{T5} dla zadania generowania zapytań SQL. Tą część treningu zakończono po 14 godzinach. Tak jak jest to ukazane na rysunku \ref{plot:resdsql-t5-accuracy} przez pierwsze 7 godzin dokładność generowanych zapytań rosła, lecz potem zaczęła regularnie spadać i w godzinie 14 osiągnęła na tyle niską wartość, że dalszego treningu zaprzestano. Podejrzewa się, że przyczyną tego jest przetrenowanie modelu, czyli zbytnie przystosowanie do danych treningowych lub zjawisko katastroficznego zapominania (ang. catastrophic forgetting) \mycite{Kirkpatrick2016}. Pierwotny model \code{T5} mógł bowiem zawierać istotną wiedzę, która została podczas dotrenowywania nadpisana. Modele są bowiem dość ograniczone pod kątem sekwencyjnej nauki zadań. Jako finalny został wybrany model z punktu w którym wartość rozważanej metryki była najwyższa.

\subsection{Wyniki}
W tabeli \ref{tab:resdsql-difficulty} przedstawione zostały wyniki przeprowadzonego eksperymentu. Rozważany model, czyli \code{RESDSQL}, podobnie jak poprzedni, generuje kompletne zapytania SQL z wartościami. Możliwe więc było obliczenie dodatkowych metryk \code{EM with values} oraz \code{EX}, co też uczyniono.

\begin{table}[H]
    \centering
    \begin{tabular}{|l|r|r|r|r|r|}
        \hline
        \thead{Zbiór} & \thead{Easy} & \thead{Medium} & \thead{Hard} & \thead{Extra} & \thead{Razem} \\
        \hline
        pol-spider & 
        \threevals{76,2}{70,6}{83,5} &
        \threevals{61,9}{56,3}{73,1} &
        \threevals{50,0}{46,0}{62,9} &
        \threevals{35,8}{30,4}{53,3} &
        \threevals{59,1}{53,8}{70,7} \\
        
        pol-spider-pl &
        \threevals{78,6}{72,6}{85,1} &
        \threevals{64,1}{58,5}{74,4} &
        \threevals{50,0}{44,8}{62,1} &
        \threevals{32,5}{27,1}{50,0} &
        \threevals{60,2}{54,5}{71,0} \\
        
        pol-spider-en &
        \threevals{73,8}{68,5}{81,9} &
        \threevals{59,6}{54,0}{71,7} &
        \threevals{50,0}{47,1}{63,8} &
        \threevals{39,2}{33,7}{56,6} &
        \threevals{58,1}{53,1}{70,4} \\
        
        en-spider &
        \threevals{81,5}{79,4}{86,3} &
        \threevals{69,3}{66,4}{75,8} &
        \threevals{51,7}{50,6}{65,5} &
        \threevals{47,0}{45,8}{50,0} &
        \threevals{65,7}{63,5}{72,4} \\
        
        \hline
        
        pol-spidersyn &
        \threevals{61,7}{57,6}{72,5} &
        \threevals{52,2}{48,7}{65,8} &
        \threevals{42,6}{41,9}{57,0} &
        \threevals{26,4}{21,9}{44,6} &
        \threevals{48,6}{45,3}{62,4} \\
        
        pol-spiderdk &
        \threevals{52,7}{48,2}{62,3} &
        \threevals{34,8}{31,5}{50,2} &
        \threevals{20,9}{20,9}{37,8} &
        \threevals{17,6}{13,3}{32,4} &
        \threevals{33,2}{29,9}{47,5} \\
        
        pol-sparc &
        \threevals{61,3}{59,0}{72,3} &
        \threevals{38,3}{33,0}{60,2} &
        \threevals{13,3}{13,3}{43,3} &
        \threevals{43,8}{43,8}{50,0} &
        \threevals{52,5}{49,5}{67,2} \\
        
        pol-cosql &
        \threevals{61,7}{56,7}{71,3} &
        \threevals{54,2}{48,3}{63,6} &
        \threevals{26,5}{25,0}{51,5} &
        \threevals{20,6}{20,6}{55,9} &
        \threevals{51,5}{47,2}{65,2} \\
        
        \hline
    \end{tabular}
    \caption{Wyniki modelu \code{RESDSQL} na poszczególnych zbiorach. Wartości w każdej komórce posiadają format EM \slashsep{ EM with values} \slashsep{ EX}}
    \label{tab:resdsql-difficulty}
\end{table}

\subsection{Analiza}
W porównaniu z dwoma wcześniej analizowanymi rozwiązaniami \code{RESDSQL} wydaje się wypadać najlepiej. Inaczej mówią jedynie wyniki dwóch testów. W przypadku zbiorów \code{pol-sparc} oraz \code{pol-spiderdk} patrząc na metrykę \code{EM} najlepiej wypadł bowiem model pierwszy, czyli \code{RAT-SQL}. W pozostałych przypadkach to jednak \code{RESDSQL} dominuje.

Warto zwrócić uwagę na fakt, że tak dobre wyniki udało się osiągnąć w stosunkowo niedługim czasie. Pomimo tego, że nauka odbywała się w sposób dwuetapowy to sumaryczny czas po którym modele osiągnęły maksymalną skuteczność okazał się dla \code{RESDSQL} najkrótszy spośród rozwiązań do tej pory rozważanych.

Tak jak zostało wcześniej zauważone, \code{RESDSQL} jest dość świeżym rozwiązaniem, które zostało opublikowane już po stworzeniu wszystkich istniejących tłumaczeń zbioru \code{Spider}, więc ich twórcy nie mieli nawet możliwości go przetestować. Autorzy metody \code{RESDSQL} kilka miesięcy od wydania swojego artykułu wyszli jednak z taką inicjatywą i postanowili dokonać treningu i testów swojego modelu na chińskim tłumaczeniu, czyli zbiorze \code{CSpider}. Dokonali tego dla wszystkich możliwych konfiguracji, czyli z wykorzystaniem modelu językowego \code{T5} we wszystkich rozmiarach oraz z użyciem techniki \code{NatSQL} i bez niej. W tabeli \ref{tab:resdsql-translations-results} zestawiono wyniki osiągnięte z wykorzystaniem modelu \code{RESDSQL} na polskim zbiorze oraz chińskim za pomocą modelu o odpowiadającej mu konfiguracji.

\begin{table}[ht!]
    \centering
    \begin{tabular}{|c|r|r|}
        \hline
        \multirow{2}{*}[-0.8em]{\thead{Tłumaczenie \\zbioru Spider}} & \multicolumn{2}{c|}{\thead{Zbiór treningowy}} \\
        \cline{2-3}
        \multirow{2}{*}{} & \thead{Tłumaczenie} & \thead{Tłumaczenie\\ + angielski} \\
        \hline
        \multicolumn{1}{|l|}{Rosyjskie (\code{PAUQ})} & \twovals{71,7}{77,9} & \twovals{---}{---} \\
        \multicolumn{1}{|l|}{Polskie (\code{Pol-Spider})} & \twovals{---}{---} & \twovals{59,1}{70,7} \\
        \hline
    \end{tabular}
    \caption{Zestawienie wyników osiągniętych przez model \code{RESDSQL} dla różnych tłumaczeń zbioru \code{Spider} i zbiorów treningowych}
    \label{tab:resdsql-translations-results}
\end{table}

 Zgodnie z zawartymi w tabeli \ref{tab:resdsql-translations-results} danymi liczbowymi na rosyjskim zbiorze udało się osiągnąć wartość metryki \code{EM} większą o ponad 10 p.p. niż dla zbioru polskiego. Do tego treningu dokonano wyłącznie na tekstach rosyjskich, bez połączenia ze zbiorem angielskim, co przypuszczalnie powinno dać jeszcze lepsze rezultaty. Duża różnica pomiędzy wynikami otrzymanymi na polskim i rosyjskim tłumaczeniu jest bardzo zastanawiająca i doszukano się dwóch możliwych jej przyczyn. Pierwsza to hipotetycznie niższy poziom skomplikowania języka rosyjskiego, co jednak jest trudnym do zweryfikowania. Drugim, zdecydowanie istotnym powodem jest fakt, że wykorzystywany w obu przypadkach wielojęzyczny model \code{mT5} był trenowany na zbiorze zawierającym większą ilość tekstów w języku rosyjskim niż polskim. Dokładny procentowy udział każdego języka w zbiorze treningowym można znaleźć w załączniku artykułu wprowadzającego model \code{mT5} \mycite{Xue2020}. Zgodnie z nim zbiór treningowy zawierał 3,71\% danych w języku rosyjskim i tylko 2,15\% w języku polskim.

\section{Model C3}
\code{C3} jest dość nowym rozwiązaniem, które zostało opublikowane w lipcu 2023 roku w artykule pod tytułem \bibtitle{C3: Zero-shot Text-to-SQL with ChatGPT} \mycite{Dong2023}. Należy ono do niedawno powstałego nurtu zaprzęgającego duże modele językowe do generowania zapytań SQL. Zapytania są kompletne, gdyż model ten przewiduje również odpowiednie wartości. Cały kod źródłowy \code{C3} został udostępniony przez autorów za pośrednictwem platformy GitHub \mycite{c3-repository}. 

\subsection{Działanie}
Sposób działania \code{C3} istotnie odbiega od funkcjonowania opisywanych wcześniej rozwiązań, gdyż należy ono do zupełnie innej kategorii. Zgodnie z wprowadzonym we wcześniejszym rozdziale podziałem nie jest to model dedykowany, lecz rozwiązanie wykorzystujące duże modele językowe wraz z techniką \code{prompt engineering}. Oznacza to, że żaden trening nie jest potrzebny, a nacisk przeniesiony został na skonstruowanie wejścia do modelu \code{GPT-3.5-Turbo}, które pozwoli jak najlepiej aktywować i wykorzystać zawartą w nim wiedzę. Dane wejściowe do tego modelu mają postać konwersacji, czyli naprzemiennie występujących wiadomości od człowieka oraz od inteligentnego asystenta. Wyjściem jest natomiast wiadomość od asystenta, będąca kontynuacją tej konwersacji.

W celu przewidzenia kompletnego zapytania SQL model \code{GPT-3.5-Turbo} jest wykorzystywany trzykrotnie. W pierwszej kolejności przekazywane jest mu pytanie wraz ze schematem bazy danych i proszony jest o zwrócenie nazw tabel posortowanych względem istotności. W drugim kroku w instrukcji wejściowej przekazywane jest pytanie, cztery najistotniejsze tabele wraz z kolumnami oraz relacje między nimi i model proszony jest o posortowanie kolumn w obrębie każdej tabeli od najbardziej do najmniej istotnej. Ostatecznie do \code{GPT-3.5-Turbo} przekazywane jest pytanie, elementy schematu uznane za istotne, klucze obce oraz wartości z bazy danych znalezione w procesie \code{schema linking} i jest on proszony o wygenerowanie gotowego zapytania.

Podczas tworzenia \code{C3} wykorzystano 3 istotne techniki, od których model wziął swoją nazwę. Określono jest bowiem jako \code{\underline{C}lear Prompting}, \code{\underline{C}alibration with Hints} oraz \code{\underline{C}onsistent Output}. Pierwsza z nich polega na tworzeniu instrukcji wejściowych o przejrzystym układzie i zamieszczaniu w nich jedynie elementów najważniejszych. Druga sprowadza się do obserwacji odchyleń względem oczekiwań w odpowiedziach modelu językowego i podawaniu we wcześniejszej konwersacji wskazówek, które to korygują. Ostatnia technika oznacza intensywne wykorzystanie strategii nazywanej \code{self-consistency} \mycite{Wang2022Consistency}. Jest ona związana z faktem, że odpowiedzi modeli językowych nie są deterministyczne -- przekazywanie tych samych informacji wejściowych zwykle skutkuje różnymi odpowiedziami. W związku z tym często dobrym pomysłem jest podanie do modelu tych samych informacji kilka razy, zebranie wszystkich odpowiedzi i wyłonienie spośród nich tej najczęstszej.

\subsection{Modyfikacje dla języka polskiego}
Z punktu widzenia przystosowania \code{C3} do języka polskiego najistotniejsza była modyfikacja promptów, czyli instrukcji wejściowych do modelu językowego. Poza tym, tak jak w przypadku \code{RAT-SQL}, należało zamienić wyrazy \code{stop words} z angielskich na polskie. Są one wykorzystywane na etapie \code{schema linking} podczas poszukiwania połączeń \code{value link}.

Modyfikacji promptów można dokonać na różne sposoby i rozważono kilka z nich. Pierwszą jest przetłumaczenie w całości na język polski. Wymaga to jednak całkowitej modyfikacji i w związku z tym wysiłek twórców \code{C3} włożony w ich dopracowanie jest w dużym stopniu tracony. Poza tym modele językowe są uczone w większości na danych angielskich i to dla nich osiągają najlepsze wyniki. Drugim z rozważanych podejść był kompletny brak tłumaczenia promptów. Nagłe pojawienie się polskiego pytania oraz elementów schematu w instrukcji wejściowej wydaje się jednak być dość nietypowe, co przypuszczalnie może pogarszać zwracane wyniki. Z wymienionych powodów zdecydowano się pozostawić prompty w języku angielskim, lecz wprowadzić w nich delikatne modyfikacje. W instrukcjach służących do znalezienia istotnych tabel i kolumn słowo \code{question} określono przymiotnikiem \code{polish}. W instrukcji służącej do finalnego generowania zapytania zrobiono to samo ze słowem \code{databases}, przetłumaczono dwa przykładowe pytania na język polski oraz dodano zdanie mówiące, że pytanie zostanie dostarczone w języku polskim. Zmodyfikowane prompty z zaznaczonymi zmianami można znaleźć w dodatku \ref{extra:c3-prompts}. Bez wątpienia ciekawym byłoby eksperymentalne zbadanie skuteczności każdego z przedstawionych sposobów modyfikacji instrukcji, lecz byłoby to również kosztowne.

\subsection{Eksperymenty}
Zmodyfikowane w powyżej opisany sposób rozwiązanie \code{C3} chciano przetestować na tych samych zbiorach, co wszystkie wcześniejsze. Problem stanowią w tym przypadku jednak koszty naliczane przez \code{OpenAI} za wykorzystanie modelu \code{GPT-3.5-Turbo}, które są znaczące. Aby temu sprostać postanowiono skonstruować mniejsze odpowiedniki posiadanych zbiorów. Poszczególne próbki wybrano przy tym tak, aby pod względem rozkładu poziomów trudności zapytań SQL zbiory mniejsze jak najdokładniej odpowiadały oryginalnym. W przypadku zbiorów zawierających elementy schematu zarówno w języku polskim i angielskim zadbano także o to, aby proporcje pomiędzy tymi częściami nie uległy zmianie. Zbiory ze schematem w jednym języku postanowiono zredukować do 50 przykładów, natomiast z dwujęzycznymi schematami do 100 przykładów. 

Wykonanie ewaluacji na wybranych pięciuset przykładach spowodowało naliczenie kwoty około 10 dolarów, czyli w przybliżeniu 40 złotych. W przeliczeniu na jedno pytanie daje to około 8 groszy. Względnie wysokie koszty są w dużym stopniu spowodowane wykorzystaniem wspomnianego mechanizmu \code{self-consistency}, który wymaga odpytywania modelu językowego wielokrotnie o to samo. Modyfikacja paru parametrów umożliwia zmniejszenie liczby tych odpytań, lecz autorzy metody pokazują w artykule, że wpływa to bezpośrednio na dokładność.

\subsection{Wyniki}
Wyniki wykonanej ewaluacji przedstawiono w tabeli \ref{tab:c3sql-difficulty}. Mają one format podobny do wcześniejszych, lecz ze względu na małą liczność zbiorów podawanie części ułamkowej poszczególnych metryk było bezzasadne. W nazwach zbiorów zawarto liczbę znajdujących się w nich przykładów.

\begin{table}[H]
    \centering


    \begin{tabular}{|l|R{0.13\textwidth}|R{0.13\textwidth}|R{0.13\textwidth}|R{0.135\textwidth}|R{0.13\textwidth}|}
        \hline
        \thead{Zbiór} & \thead{Easy} & \thead{Medium} & \thead{Hard} & \thead{Extra Hard} & \thead{Razem} \\
        \hline
        Pol-Spider-100 & 
        \threevalsx{54}{54}{83} &
        \threevalsx{43}{43}{91} &
        \threevalsx{44}{38}{44} &
        \threevalsx{13}{13}{75} &
        \threevalsx{41}{40}{79} \\
        
        Pol-Spider-PL-50 &
        \threevalsx{58}{58}{83} &
        \threevalsx{50}{50}{91} &
        \threevalsx{38}{38}{63} &
        \threevalsx{13}{13}{75} &
        \threevalsx{44}{44}{82} \\
        
        Pol-Spider-EN-50 &
        \threevalsx{50}{50}{83} &
        \threevalsx{36}{36}{91} &
        \threevalsx{50}{38}{25} &
        \threevalsx{13}{13}{75} &
        \threevalsx{38}{36}{76} \\
        
        \hline
        
        Pol-Spidersyn-100 &
        \threevalsx{32}{32}{55} &
        \threevalsx{39}{36}{61} &
        \threevalsx{28}{27}{56} &
        \threevalsx{\s0}{\s0}{38} &
        \threevalsx{29}{28}{55} \\
        
        Pol-Spiderdk-100 &
        \threevalsx{100}{75}{75} &
        \threevalsx{50}{48}{59} &
        \threevalsx{36}{36}{79} &
        \threevalsx{15}{15}{50} &
        \threevalsx{51}{45}{63} \\
        
        Pol-Sparc-100 &
        \threevalsx{61}{61}{79} &
        \threevalsx{22}{18}{68} &
        \threevalsx{\s0}{\s0}{50} &
        \threevalsx{\s0}{\s0}{\s0} &
        \threevalsx{46}{45}{73} \\
        
        Pol-Cosql-100 &
        \threevalsx{62}{56}{87} &
        \threevalsx{31}{31}{77} &
        \threevalsx{29}{22}{36} &
        \threevalsx{\s0}{\s0}{50} &
        \threevalsx{44}{40}{74} \\
        
        \hline
    \end{tabular}
    \lcaption{Wyniki modelu \code{C3} na poszczególnych zbiorach}{Wartości w każdej komórce posiadają format \mbox{EM Without Values / EM / EX} i zostały wyrażone w procentach.}
\label{tab:c3sql-difficulty}
    
    
\end{table}

\subsection{Analiza}
Zgodnie z rankingiem zbioru \code{Spider} analizowane teraz rozwiązanie jest spośród wszystkich w tej pracy omawianych najlepsze pod kątem metryki \code{EX}. Przeprowadzone eksperymenty to potwierdzają. Najlepsze wyniki nie zostały osiągnięte jedynie w przypadku zbioru \code{Pol-Spidersyn}, lecz również są dość wysokie.

Wartość metryki \code{EM} osiągnęła z drugiej strony zaskakująco niską wartość. Pod jej kątem \code{C3} wydaje się być najgorszym z analizowanych rozwiązań. Skąd wynika tak duża rozbieżność? Powodem jest prawdopodobnie fakt, iż metryka \code{EM} dokonuje strukturalnego porównania wzorcowych i przewidzianych  zapytań, a ten sam cel może być realizowany przez dwa zapytania całkowicie różne. W przypadku wcześniejszych rozwiązań problem ten nie był tak widoczny, ponieważ były one uczone na części treningowej i w związku z tym nauczyły się bardziej preferować pewne sposoby generowania zapytań od innych. \code{C3} produkuje za to zapytania w sposób znacznie bardziej swobodny, co powoduje problemy ze wstrzeleniem się w oczekiwaną odpowiedź. Nie oznacza to jednak, że jest ona błędna. 

Trzeba koniecznie zwrócić uwagę na to, że mimo ogólnie zaniżonej metryki \code{EM} udało się osiągnąć najlepszy dotychczas wynik na zbiorze \code{Pol-Spiderdk}, który zawiera wiedzę domenową. \code{RESDSQL} przebija dotychczas najlepszy model \code{RAT-SQL} o druzgoczące 10 punktów procentowych. Jest to uzasadnione, ponieważ wewnętrznie wykorzystywany model \code{GPT-3.5-Turbo} został nauczony wcześniej na ogromnym zbiorze danych i zyskał przez to specjalistyczną wiedzę w bardzo wielu dziedzinach.

Słusznie może pojawić się obawa, że otrzymywane z wykorzystaniem \code{C3} wyniki ewaluacji nie są wiarygodne. Przypuszczalnie model językowy \code{GPT-3.5-Turbo} mógł bowiem być trenowany na wykorzystywanych podczas ewaluacji danych lub im podobnych i w związku z tym osiągać w testach zawyżone wyniki. Sytuacja taka nazywana jest wyciekiem danych. Zbiory treningowe nie mogły zawierać przykładów identycznych, gdyż żadne polskie tłumaczenia zbioru \code{Spider} wcześniej nie istniały. Artykuł \bibtitle{Rethinking Benchmark and Contamination for Language Models with Rephrased Samples} \mycite{Yang2023} wskazuje jednak, że kontaminacja może zostać wprowadzona do zbioru treningowego także poprzez tłumaczenia na inne języki. Z tego powodu angielski zbiór \code{Spider} oraz jego istniejące już tłumaczenia mogą stanowić zagrożenie dla rzetelności testów.

Jeśli chodzi o ranking zbioru \code{Spider}, to takiej obawy nie ma, ponieważ część testowa jest utajona i nie ma możliwości, aby \code{OpenAI}, nawet przez przypadek, dopuściło do dołączenia tych danych do zbioru treningowego. W niniejszej pracy, jak też wielu innych, z powodu braku dostępności części testowej jej rolę przejmuję publicznie dostępna część pierwotnie nazwana walidacyjną, więc obawa o wyciek danych staje się realna. Zgodnie z artykułem wprowadzającym model \mbox{\code{GPT-3}} \mycite{Brown2020} \code{OpenAI} dokonuje usuwania ze zbioru treningowego danych pokrywających się z wykorzystywanymi przez nich zbiorami testowymi. Swojego modelu nie testowali na zbiorze \code{Spider}, więc prawdopodobnie specjalne wysiłki nie zostały podjęte w celu wykluczenia tych danych z treningu. Trzeba się więc liczyć z tym, że dane przedstawione w powyższej tabeli mogą być zawyżone.

W przypadku \code{C3} należy zaakcentować dość długi czas potrzebny na wygenerowanie odpowiedzi na dostarczone pytania, ponieważ dla każdego wymagane jest do 30 sekund oczekiwania. Inną zauważoną wadą jest bardzo duża zależność od \code{OpenAI}. Wiąże się to oczywiście z kosztami, ale wprowadza również komplikacje w kwestii utrzymania działania całego rozwiązania w dłuższej perspektywie, gdyż \code{OpenAI} może dowolnie zmieniać cenniki, usuwać stare modele, czy modyfikować interfejsy. Nie posiada się już kontroli nad całym systemem.
