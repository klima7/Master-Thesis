\section{RAT-SQL}
\code{RAT-SQL} to bardzo rozpoznawalny model, który był krzyżowany na przestrzeni czasu z różnymi innymi algorytmami i jego warianty można znaleźć w rankingu zbioru \code{Spider} na wielu pozycjach. Rozwiązanie to zostało zaproponowane w 2021 roku przez Bailin Wang oraz Richarda Shin \cite{Wang2019}. Doczekało się dwóch kolejnych iteracji, określanych jako \code{RAT-SQL v2} oraz \code{RAT-SQL v3}. W niniejszej pracy rozważana jest ta ostatnia, której kod dostępny jest na platformie Github w repozytorium Microsoftu \cite{ratsql-repository}. Model ten produkuje jednak zapytania bez wartości, co ogranicza jego praktyczne zastosowanie.

\subsection{Działanie}
Nazwa modelu \code{RAT-SQL} pochodzi od angielskiego wyrażenia "Relation Aware Transformer", które doskonale opisuje to, co jest w tym rozwiązaniu najważniejsze. Wykorzystuję ono bowiem enkodowanie oparte na grafie, które zostało zrealizowane przy pomocy sieci typu transformer. Standardowo sieci takie potrafią analizować jedynie sekwencje, więc konieczne było zmodyfikowanie ich w ten sposób, aby działały również dla grafów. Dzięki dokonanym zmianom stały się one świadome relacji pomiędzy poszczególnymi elementami sekwencji.

Wszystkie sieci typu transformer posiadają element odpowiedzialny za uczenie się powiązań pomiędzy elementami wejściowymi w postaci mechanizmu uwagi (ang. attention). Autorzy \code{RAT-SQL} zbadali eksperymentalnie działanie tego mechanizmu dla problemu generowania zapytań SQL i zauważyli, że znajdywane połączenia są jednak dość słabe. Postanowili więc zmodyfikować mechanizm uwagi tak, aby poza znajdywaniem powiązań w standardowy, rozmyty sposób można było jako dodatkowe wejście podać powiązania z góry znane. 

Relacje, które postanowiono jawnie zakodować poprzez zmodyfikowany mechanizm uwagi to przede wszystkim przynależność poszczególnych kolumn do tabel i powiązania tworzone przez klucze obce. Wykorzystano również etap \code{schema linking} w celu znalezienia połączeń pomiędzy fragmentami pytania i elementami bazy danych. Zrealizowano to poprzez wydobycie z pytania wszystkich n-gramów o długości od 1 do 5 i wyszukane ich dokładnych lub częściowych dopasowań wśród nazw tabel i kolumn. Połączenia typu \code{value link} znaleziono natomiast poprzez rozważenie każdego słowa z pytania osobno i wyszukiwanie go w każdej z dostępnych kolumn. Wszystkie te relacje tworzą skomplikowany graf, który \code{RAT-SQL} pozwala zakodować.

Przed przekazaniem wszystkich informacji wejściowych do transformera dokonującego enkodowania konieczna jest zamiana wartości tekstowych do postaci wektorowej. Uwaga ta tyczy się nazw tabel i kolumn oraz pytania. \code{RAT-SQL} pozwala dokonać takiej konwersji na dwa sposoby: z wykorzystaniem gotowych embeddingów uzyskanych metodą \code{GloVe} lub pretrenowanym modelem \code{BERT}. Po enkodowaniu musi nastąpić dekodowanie, w którym wykorzystano rekurencyjne komórki \code{LSTM} do generowania akcji pozwalających zbudować drzewo \code{AST}.



\subsection{Modyfikacje dla języka polskiego}

\subsection{Eksperymenty z wynikami}
