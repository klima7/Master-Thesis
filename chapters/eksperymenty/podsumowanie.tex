\section{Podsumowanie wyników}
W czterech poprzednich podrozdziałach przedstawione zostały wyniki osiągnięte przez cztery modele. Te rezultaty w tabelarycznej formie, pomimo dostarczania wielu informacji, nie są wygodne do porównywania ze sobą. Zauważono, że wyniki metryk dla wszystkich modeli i zbiorów można wygodnie przedstawić za pomocą wykresów radarowych. Umieszczono je na rysunkach \ref{fig:experiments-em} oraz \ref{fig:experiments-ex}, odpowiednio dla metryki \code{EM Without Values} oraz \code{EX}.

Zgodnie z wykresem pokazującym wartości metryki \code{EM} najlepiej działającym modelem zdaje się być \code{RESDSQL}. Zgodnie z drugim wykresem najwyższą skuteczności wykazuje za to bezsprzecznie model \code{C3}, a \code{RESDSQL} jest na drugiej pozycji. Tak więc te dwa rozwiązania wydają się najbardziej obiecujące i w związku z tym zostaną w kolejnym rozdziale zintegrowane z interfejsem graficznym i poddane dalszym testom.

\begin{figure}[H]
  \centering
  \includesvg[width=0.9\linewidth]{images/experiments_em}
  \caption{Wyniki metryki \code{EM Without Values} dla wszystkich modeli i zbiorów}
  \label{fig:experiments-em}
\end{figure}

\begin{figure}[H]
  \centering
  \includesvg[width=0.9\linewidth]{images/experiments_ex}
  \lcaption{Wyniki metryki \code{EX} dla wszystkich modeli i zbiorów}{Dla modelu \code{RAT-SQL} metryki \code{EX} nie da się obliczyć.}
  \label{fig:experiments-ex}
\end{figure}
