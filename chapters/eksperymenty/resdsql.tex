\section{Model RESDSQL}
\code{RESDSQL} to dość nowe rozwiązanie, gdyż powstało na początku 2023 roku. Zostało opublikowane w artykule \bibtitle{RESDSQL: Decoupling Schema Linking and Skeleton Parsing for Text-to-SQL} \mycite{Li2023resdsql} przez Haoyang Li i innych. Na chwilę pisania niniejszej pracy jest to najwyżej znajdujące się w rankingu \code{Spider} rozwiązanie, które nie opiera się na wykorzystaniu dużych pretrenowanych modeli językowych od \code{OpenAI} i do którego dostępny jest kod źródłowy \mycite{resdsql-repository}.

\subsection{Działanie}
\code{RESDSQL} bazuje w dużej mierzę na pretrenownym modelu językowym \code{T5} \mycite{Raffel2019}. Jest on kompletnym modelem typu transformer, który służy do generowania tekstowych odpowiedzi na podstawie tekstowych informacji wejściowych. Został już wcześniej wytrenowany na obszernym zbiorze, więc zawiera dużą ilość wiedzy. \code{RESDSQL} dokonuje dotrenowania tego modelu tak, aby zwracał zapytania SQL będące odpowiedzią na podawane pytania. Podczas konstruowania sekwencji wejściowej oraz dekodowania sekwencji wyjściowej wprowadza jednak ciekawe techniki, które poprawiają jego skuteczność dla rozważanego problemu \code{Text-to-SQL}. Ich głównym celem jest rozdzielenie od siebie generowania szkieletu zapytań i uzupełniania go konkretnymi nazwami tabel i kolumn, co z resztą zostało podkreślone w tytule artykułu.

Ważnym elementem, odróżniającym \code{RESDSQL} od wcześniej opisanych rozwiązań jest to, że do enkodera zawartego wewnątrz \code{T5} nie są przekazywane wszystkie nazwy tabel i kolumn, lecz tylko te, które zostały uznane w kontekście danego pytania za najbardziej istotne. Celem tego działania jest odciążenie enkodera z wykonywania skomplikowanego procesu \code{schema linking}. Aby dostać informację o istotności poszczególnych tabel i kolumn dla podanego pytania trenowana jest wcześniej całkowicie niezależna sieć, która przyjmuję pytanie oraz wszystkie nazwy tabel i kolumn i dla każdej z nich zwraca prawdopodobieństwo, które może być interpretowane jako stopień istotności. Sieć ta określana jest przez swoich autorów mianem \code{cross-encoder}. Ostatecznie do modelu \code{T5} przekazywane są 4 najistotniejsze tabele, a dla każdej z nich 5 najważniejszych kolumn.

Z punktu widzenia dekodowania największą nowością jest to, że model \code{T5} nie jest dotrenowywany tak, aby od razu produkować kompletne zapytania, lecz na początku zwrócić szkielet, a dopiero za nim wykonywalne zapytanie. Uzasadnieniem opisanego zachowania jest to, że początkowe przewidzenie szkieletu jest w miarę prostym problemem, więc powinno mieć dużą dokładność. Dekodery w modelach językowych posiadają własność nazywaną autoregresyjnością, która polega na tym, że podczas generowania kolejnych fragmentów odpowiedzi brany jest pod uwagę tekst wygenerowany do tej pory. Dzięki temu model \code{T5} podczas produkowania drugiej części odpowiedzi, która zawiera kompletne zapytanie SQL, może odwoływać się do wcześniejszej części z szablonem i traktować ją jako pewnego rodzaju notatnik.

Ciekawą techniką, z którą \code{RESDSQL} można łączyć, jest \code{NatSQL} \mycite{Gan2021natsql}. Wprowadza ona alternatywną reprezentację dla zapytań SQL, która bardziej przypomina język naturalny i w związku z tym jest łatwiejsza do nauczenia i generowania przez większość modeli. Jednocześnie jest ona na tyle jednoznaczna, że można przekonwertować ją na tradycyjne zapytania SQL. Zgodnie z eksperymentami autorów \code{RESDSQL} zastąpienie tradycyjnych zapytań za pomocą \code{NatSQL} pozwoliło na zbiorze \code{Spider} zwiększyć skuteczności mierzoną metryką \code{EM} o około 2 punkty procentowe.

\subsection{Modyfikacje dla języka polskiego}
Rozwiązanie \code{RESDSQL} okazało się wyjątkowo proste do przystosowania dla języka polskiego, bo nie trzeba było wykonywać żadnych kreatywnych modyfikacji. Metoda ta została bowiem już wcześniej wykorzystana do pracy na rosyjskim zbiorze \code{PAUQ} i stosowny kod znajdywał się w repozytorium. Wystarczyło dodać kilka nowych skryptów, które nieznacznie różnią się od istniejących.

Najbardziej istotną modyfikacją, niezbędną do nauki na polskim, czy też rosyjskim tłumaczeniu, jest zmiana wykorzystywanego modelu \code{T5} na \code{mT5} \mycite{Xue2020}. Różnica pomiędzy nimi jest jedynie taka, że pierwszy został nauczony na tekstach angielskich, natomiast drugi na zbiorze zawierającym 101 różnych języków, w tym wymienione dwa.

Okazuję się, że niestety dla języka polskiego nie można łatwo wykorzystać obiecującego połączenia niniejszej metody z \code{NatSQL}. Przyczyną jest brak udostępnienia przez autorów \code{NatSQL} skryptu przekształcającego tradycyjne zapytania SQL do zaprojektowanej przez nich postaci. Opublikowane są jedynie zapytania ze zbioru \code{Spider} po dokonaniu takiej konwersji. W przypadku rosyjskiego tłumaczenia udało się zastosować \code{NatSQL}, lecz wymagało to przekształcenia każdego zapytania w sposób ręczny. Dla zbioru polskiego należałoby postąpić analogicznie, co wymagałoby wiele żmudnej pracy, której postanowiono uniknąć.

\subsection{Eksperymenty}
W ramach eksperymentu postanowiono dokonać nauki \code{RESDSQL} na zbiorze \code{Mix-Spider}, ponieważ, jak wcześniej zauważono, nauka na dwujęzycznych zbiorach wydaje się mieć lepsze efekty niż na jednojęzycznych. Konieczny był wybór konkretnego wariantu dotrenowywanego modelu \code{mT5}, ponieważ występuje on w kilku rozmiarach. Na mocniejszej z dwóch posiadanych kart graficznych, czyli \code{RTX 3080}, udało się uruchomić jedynie wariant najmniejszy -- \code{mt5-base}. Sprawdzając poszczególne rozmiary, stosowano minimalną wielkość wsadu (ang. batch size), czyli ilość danych jednocześnie przekazywanych do modelu, aby zmniejszyć zapotrzebowanie na pamięć \code{VRAM}. Po dokonaniu głębszej ingerencji w procedurę treningu udało się ostatecznie uruchomić ją także dla większego modelu. Wymagało to jednak załadowania wag i treningu z wykorzystaniem arytmetyki 16-bitowej zamiast powszechnie stosowanej 32-bitowej. Może się to wiązać z pojawieniem podczas nauki różnych niestabilności, więc z powodu braku doświadczenia z tego typu treningiem postanowiono pozostać przy oryginalnym.

Zgodnie z wcześniejszym opisem wykorzystanie \code{RESDSQL} wymaga w pierwszej kolejności wytrenowania sieci \code{cross-encoder}, która dokonuje oceny istotności tabel i kolumn. Trening ten wykonywano przez 4 godziny, a wykres przedstawiający zmianę skuteczności modelu w jego trakcie umieszczono na rysunku \ref{plot:resdsql-classifier-accuracy}. Wspomniana skuteczność jest tutaj wyrażana za pomocą metryki \code{AUC} \mycite{Ling2003}, spotykanej w zadaniach klasyfikacji. Po 4 godzinach trening został automatycznie przerwany, ponieważ od dłuższego czasu nie nastąpiło zwiększenie skuteczności rozumianej jako sumy \code{AUC} dla tabel i kolumn. Jako produkt tego etapu został wybrany model z zestawem wag, dla którego metryka ta była najwyższa.

\begin{figure}[ht!]
\centering
\begin{subfigure}{0.49\textwidth}
    \begin{tikzpicture}
      \begin{axis}[
        width=\linewidth,
        height=\linewidth,
        xmin=0, xmax=4.1,
        grid=major,
        xlabel={Czas treningu [godziny]},
        ylabel=AUC,
        legend cell align={left},
        legend pos=south east,
      ]
        \addplot table[x=time,y=table,col sep=comma] {plots/resdsql_classifier_training.csv}; 
        \addplot table[x=time,y=column,col sep=comma] {plots/resdsql_classifier_training.csv}; 
        \legend{dla tabel, dla kolumn}
      \end{axis}
    \end{tikzpicture}
    \caption{model oceniający istotność tabel i kolumn}
    \label{plot:resdsql-classifier-accuracy}
\end{subfigure}
\hfill
\begin{subfigure}{0.49\textwidth}
    \begin{tikzpicture}
      \begin{axis}[
        width=\linewidth,
        height=\linewidth,
        xmin=0, xmax=14,
        grid=major,
        xlabel={Czas treningu [godziny]},
        ylabel={EM without values [\%]},
      ]
        \addplot table[x=step,y expr=\thisrow{em} * 100,col sep=comma] {plots/resdsql_t5_training.csv}; 
      \end{axis}
    \end{tikzpicture}
    \caption{model przewidujący zapytania}
    \label{plot:resdsql-t5-accuracy}
\end{subfigure}
\lcaption{Wyniki modeli składowych \code{RESDSQL} w czasie trwania treningu}{Modele sprawdzano na części testowej zbioru \code{Pol-Spider}.}
\label{plot:resdsql-accuracy}
\end{figure}

Drugim etapem nauki było dotrenowywanie modelu \code{T5} dla zadania generowania zapytań SQL. Tę część treningu zakończono po 14 godzinach. Tak jak jest to ukazane na rysunku \ref{plot:resdsql-t5-accuracy} przez pierwsze 7 godzin dokładność generowanych zapytań rosła, lecz potem zaczęła regularnie spadać i w godzinie 14 osiągnęła na tyle niską wartość, że dalszego treningu zaprzestano. Przyczyną tego mogło być przetrenowanie modelu \mycite{overfitting2004}, czyli zbytnie przystosowanie do danych treningowych. Jako finalny został wybrany model z punktu, w którym wartość rozważanej metryki była najwyższa.

\subsection{Wyniki}
W tabeli \ref{tab:resdsql-difficulty} przedstawione zostały wyniki przeprowadzonego eksperymentu. Rozważany model, czyli \code{RESDSQL}, podobnie jak poprzedni, generuje kompletne zapytania SQL z wartościami. Możliwe więc było obliczenie metryk \code{EM} oraz \code{EX}.

\begin{table}[H]
    \centering
    \begin{tabular}{|l|r|r|r|r|r|}
        \hline
        \thead{Zbiór} & \thead{Easy} & \thead{Medium} & \thead{Hard} & \thead{Extra Hard} & \thead{Razem} \\
        \hline
        Pol-Spider & 
        \threevals{76,2}{70,6}{83,5} &
        \threevals{61,9}{56,3}{73,1} &
        \threevals{50,0}{46,0}{62,9} &
        \threevals{35,8}{30,4}{53,3} &
        \threevals{59,1}{53,8}{70,7} \\
        
        Pol-Spider-PL &
        \threevals{78,6}{72,6}{85,1} &
        \threevals{64,1}{58,5}{74,4} &
        \threevals{50,0}{44,8}{62,1} &
        \threevals{32,5}{27,1}{50,0} &
        \threevals{60,2}{54,5}{71,0} \\
        
        Pol-Spider-EN &
        \threevals{73,8}{68,5}{81,9} &
        \threevals{59,6}{54,0}{71,7} &
        \threevals{50,0}{47,1}{63,8} &
        \threevals{39,2}{33,7}{56,6} &
        \threevals{58,1}{53,1}{70,4} \\
        
        En-Spider &
        \threevals{81,5}{79,4}{86,3} &
        \threevals{69,3}{66,4}{75,8} &
        \threevals{51,7}{50,6}{65,5} &
        \threevals{47,0}{45,8}{50,0} &
        \threevals{65,7}{63,5}{72,4} \\
        
        \hline
        
        Pol-Spidersyn &
        \threevals{61,7}{57,6}{72,5} &
        \threevals{52,2}{48,7}{65,8} &
        \threevals{42,6}{41,9}{57,0} &
        \threevals{26,4}{21,9}{44,6} &
        \threevals{48,6}{45,3}{62,4} \\
        
        Pol-Spiderdk &
        \threevals{52,7}{48,2}{62,3} &
        \threevals{34,8}{31,5}{50,2} &
        \threevals{20,9}{20,9}{37,8} &
        \threevals{17,6}{13,3}{32,4} &
        \threevals{33,2}{29,9}{47,5} \\
        
        Pol-Sparc &
        \threevals{61,3}{59,0}{72,3} &
        \threevals{38,3}{33,0}{60,2} &
        \threevals{13,3}{13,3}{43,3} &
        \threevals{43,8}{43,8}{50,0} &
        \threevals{52,5}{49,5}{67,2} \\
        
        Pol-Cosql &
        \threevals{61,7}{56,7}{71,3} &
        \threevals{54,2}{48,3}{63,6} &
        \threevals{26,5}{25,0}{51,5} &
        \threevals{20,6}{20,6}{55,9} &
        \threevals{51,5}{47,2}{65,2} \\
        
        \hline
    \end{tabular}
    \lcaption{Wyniki modelu \code{RESDSQL} na poszczególnych zbiorach}{Wartości w każdej komórce posiadają format \mbox{EM Without Values / EM / EX} i zostały wyrażone w procentach.}
    \label{tab:resdsql-difficulty}
\end{table}

\subsection{Analiza}
W porównaniu z dwoma wcześniej analizowanymi rozwiązaniami \code{RESDSQL} wydaje się wypadać najlepiej. Inaczej mówią jedynie wyniki dwóch testów. W przypadku zbiorów \code{Pol-Sparc} oraz \code{Pol-Spiderdk}, patrząc na metrykę \code{EM}, najlepiej wypadł bowiem model pierwszy, czyli \code{RAT-SQL}. W pozostałych przypadkach to jednak \code{RESDSQL} dominuje.

Warto zwrócić uwagę na fakt, że tak dobre wyniki udało się osiągnąć w stosunkowo niedługim czasie. Pomimo tego, że nauka odbywała się w sposób dwuetapowy to sumaryczny czas, po którym modele osiągnęły maksymalną skuteczność okazał się dla \code{RESDSQL} najkrótszy spośród rozwiązań do tej pory rozważanych.

Tak jak zostało wcześniej zauważone, \code{RESDSQL} jest dość świeżym rozwiązaniem, które zostało opublikowane już po stworzeniu wszystkich istniejących tłumaczeń zbioru \code{Spider}, więc ich twórcy nie mieli nawet możliwości wykonać na \code{RESDSQL} eksperymentów. Autorzy tej metody jakiś czas od wydania swojego artykułu wyszli jednak z inicjatywą i postanowili dokonać treningu i testów swojego modelu na chińskim tłumaczeniu, czyli zbiorze \code{CSpider}. Wyniki zamieścili w repozytorium na platformie \code{GitHub}. W tabeli \ref{tab:resdsql-translations-results} zestawiono wyniki osiągnięte z wykorzystaniem modelu \code{RESDSQL} na zbiorze polskim i chińskim. Na tym ostatnim wytrenowanych zostało kilka modeli, więc wybrano ten, który odpowiada polskiemu pod względem konfiguracji.

\begin{table}[ht!]
    \centering
    \begin{tabular}{|c|r|r|}
        \hline
        \multirow{2}{*}[-0.8em]{\thead{Tłumaczenie \\zbioru Spider}} & \multicolumn{2}{c|}{\thead{Zbiór treningowy}} \\
        \cline{2-3}
        \multirow{2}{*}{} & \thead{Tłumaczenie} & \thead{Tłumaczenie\\ + angielski} \\
        \hline
        \multicolumn{1}{|l|}{Chińskie (\code{CSpider})} & \twovals{71,7}{77,9} & \twovals{\varendash[20pt]}{\varendash[20pt]} \\
        \multicolumn{1}{|l|}{Polskie (\code{Pol-Spider})} & \twovals{\varendash[20pt]}{\varendash[20pt]} & \twovals{59,1}{70,7} \\
        \hline
    \end{tabular}
    \lcaption{Zestawienie wyników modelu \code{RESDSQL} dla polskiego i chińskiego tłumaczenia}{Komórki tabeli zawierają wyniki metryk w formacie EM Without Values / EX, podane procentowo.}
    \label{tab:resdsql-translations-results}
\end{table}

 Zgodnie z zawartymi w tabeli \ref{tab:resdsql-translations-results} danymi liczbowymi na rosyjskim zbiorze udało się osiągnąć wartość metryki \code{EM Without Values} większą o ponad 10 punktów procentowych niż dla zbioru polskiego. Do tego treningu dokonano wyłącznie na tekstach chińskich, bez dołączania zbioru angielskiego, co przypuszczalnie powinno dać jeszcze lepsze rezultaty. Duża różnica pomiędzy wynikami otrzymanymi na polskim i rosyjskim tłumaczeniu jest dość zastanawiająca i pierwszym przypuszczeniem było to, że wykorzystywany w obu przypadkach wielojęzyczny model \code{mT5} był trenowany na zbiorze zawierającym większą ilość tekstów w języku chińskim niż polskim. Załącznik do artykułu wprowadzającego \code{mT5} \mycite{Xue2020} ujawnia jednak, że było wręcz odwrotnie -- w zbiorze treningowym znalazło się więcej danych polskich niż chińskich. Przypuszczenie to zostało więc odrzucone. Innym powodem mogła być lepsza jakoś zbioru chińskiego, który został przetłumaczony manualnie, co ułatwiło modelowi naukę. 
 