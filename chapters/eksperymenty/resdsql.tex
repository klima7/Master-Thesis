\section{RESDSQL}
\code{RESDSQL} to dość nowe rozwiązanie, gdyż powstało na początku 2023 roku. Zostało opublikowane w artykule \bibtitle{RESDSQL: Decoupling Schema Linking and Skeleton Parsing for Text-to-SQL} \cite{Li2023resdsql} przez Haoyang Li i innych. Na chwilę pisania niniejszej pracy jest to najwyżej znajdujące się w rankingu \code{Spider} rozwiązanie, które nie opiera się na wykorzystaniu dużych pretrenowanych modeli językowych od \code{OpenAI} i do którego dostępny jest kod źródłowy \cite{resdsql-repository}.

\subsection{Działanie}

\subsection{Modyfikacje dla języka polskiego}

\subsection{Eksperymenty}

\begin{figure}[ht!]
  \begin{center}
    \begin{tikzpicture}
      \begin{axis}[
        width=\linewidth,
        height=\fpeval{0.5*\linewidth},
        xmin=0, xmax=4.1,
        grid=major,
        xlabel={Czas treningu [godziny]},
        ylabel=AUC,
        legend cell align={left},
        legend pos=south east,
      ]
        \addplot table[x=time,y=table,col sep=comma] {plots/resdsql_classifier_training.csv}; 
        \addplot table[x=time,y=column,col sep=comma] {plots/resdsql_classifier_training.csv}; 
        \legend{AUC dla tabel, AUC dla kolumn}
      \end{axis}
    \end{tikzpicture}
    \caption{RESDSQL Klasyfikator}
    \label{plot:resdsql-classifier-accuracy}
  \end{center}
\end{figure}

\begin{figure}[ht!]
  \begin{center}
    \begin{tikzpicture}
      \begin{axis}[
        width=\linewidth,
        height=\fpeval{0.5*\linewidth},
        xmin=0, xmax=15,
        grid=major,
        xlabel={Czas treningu [godziny]},
        ylabel=EM without values,
      ]
        \addplot table[x=step,y=em,col sep=comma] {plots/resdsql_t5_training.csv}; 
      \end{axis}
    \end{tikzpicture}
    \caption{Wyniki metryki \code{EM} na części testowej zbioru \code{pol-spider} w czasie trwania treningu modelu}
    \label{plot:resdsql-accuracy}
  \end{center}
\end{figure}

\subsection{Wyniki}

\newcommand{\slashsep}{\textcolor{red}{\textbf{/}}}
\newcommand{\threevals}[3]{\small{#1}\slashsep\small{#2}\slashsep\small{#3}}

\begin{table}[H]
    \centering
    \begin{tabular}{|l|r|r|r|r|r|}
        \hline
        \thead{Zbiór} & \thead{Easy} & \thead{Medium} & \thead{Hard} & \thead{Extra} & \thead{Razem} \\
        \hline
        pol-spider & 
        \threevals{76,2}{70,6}{83,5} &
        \threevals{61,9}{56,3}{73,1} &
        \threevals{50,0}{46,0}{62,9} &
        \threevals{35,8}{30,4}{53,3} &
        \threevals{59,1}{53,8}{70,7} \\
        
        pol-spider-pl &
        \threevals{78,6}{72,6}{85,1} &
        \threevals{64,1}{58,5}{74,4} &
        \threevals{50,0}{44,8}{62,1} &
        \threevals{32,5}{27,1}{50,0} &
        \threevals{60,2}{54,5}{71,0} \\
        
        pol-spider-en &
        \threevals{73,8}{68,5}{81,9} &
        \threevals{59,6}{54,0}{71,7} &
        \threevals{50,0}{47,1}{63,8} &
        \threevals{39,2}{33,7}{56,6} &
        \threevals{58,1}{53,1}{70,4} \\
        
        en-spider &
        \threevals{81,5}{79,4}{86,3} &
        \threevals{69,3}{66,4}{75,8} &
        \threevals{51,7}{50,6}{65,5} &
        \threevals{47,0}{45,8}{50,0} &
        \threevals{65,7}{63,5}{72,4} \\
        
        \hline
        
        pol-spidersyn &
        \threevals{61,7}{57,6}{72,5} &
        \threevals{52,2}{48,7}{65,8} &
        \threevals{42,6}{41,9}{57,0} &
        \threevals{26,4}{21,9}{44,6} &
        \threevals{48,6}{45,3}{62,4} \\
        
        pol-spiderdk &
        \threevals{52,7}{48,2}{62,3} &
        \threevals{34,8}{31,5}{50,2} &
        \threevals{20,9}{20,9}{37,8} &
        \threevals{17,6}{13,3}{32,4} &
        \threevals{33,2}{29,9}{47,5} \\
        
        pol-sparc &
        \threevals{61,3}{59,0}{72,3} &
        \threevals{38,3}{33,0}{60,2} &
        \threevals{13,3}{13,3}{43,3} &
        \threevals{43,8}{43,8}{50,0} &
        \threevals{52,5}{49,5}{67,2} \\
        
        pol-cosql &
        \threevals{61,7}{56,7}{71,3} &
        \threevals{54,2}{48,3}{63,6} &
        \threevals{26,5}{25,0}{51,5} &
        \threevals{20,6}{20,6}{55,9} &
        \threevals{51,5}{47,2}{65,2} \\
        
        \hline
    \end{tabular}
    \caption{...}
    \label{tab:resdsql-difficulty}
\end{table}

\subsection{Analiza}